
% This LaTeX was auto-generated from MATLAB code.
% To make changes, update the MATLAB code and republish this document.

\documentclass{article}
\usepackage{graphicx}
\usepackage{color}

\sloppy
\definecolor{lightgray}{gray}{0.5}
\setlength{\parindent}{0pt}

\begin{document}

    
    
\section*{REFERENCES}

\begin{par}
 \small \parskip=2pt
\def\parr{{\tiny\sl ~CHECK}\par}
Each reference is followed by a note highlighting a
contribution of that publication that is relevant to this book.
These notes are by no means comprehensive: in most cases the references
include other significant contributions too.
Papers listed by authors such as Cauchy,
Chebyshev, Gauss, Jacobi, and Weierstrass
can also be found in their collected works.
\par
Among mathematicians of the 19th century, it is hard not to be struck
by the remarkable creativity of Jacobi (1804--1851),
who in his short life made key early contributions to
barycentric interpolation [1825], orthogonal polynomials and
Gauss quadrature [1826], and
Pad\'e approximation and rational interpolation [1846], as well
as innumerable topics outside the scope of this book.
\par
As of May 2012, the Mathematics Genealogy Project lists
8605 adademic descendents of Pafnuty Lvovich Chebyshev.  For
example, one chain runs
Chebyshev--Lyapunov--Steklov--Smirnov--Sobolev--V. I. Lebedev,
and another runs
Chebyshev--Markov--Voronoy--Sierpinsky--Mazurkiewicz--Zygmund--Stein--C. Fefferman.
\par
{\sc N. I. Achieser,} On extremal properties of certain rational
functions (Russian), {\em DAN} 18 (1930), 495--499.
[Equioscillation characterization for best rational approximations.]
\par
{\sc N. I. Achieser,} {\em Theory of Approximation,} Dover, 1992.
[Treatise by one of Chebyshev's academic great-grandsons, first published
in 1956.]
\par
{\sc V. Adamyan, D. Arov and M. Krein,} Analytic properties of
Schmidt pairs for a Hankel operator and the generalized
Schur--Takagi problem, {\em Math.\ USSR Sb.}\ 15 (1971), 31--73.
[Major work with a general extension of results of
Carath\'eodory, Fej\'er, Schur and Takagi
to rational approximation on the unit circle.]
\par
{\sc L. Ahlfors,} {\em Complex Analysis,} 3rd ed., McGraw-Hill, 1978.
[A terse and beautiful complex analysis text by one of the masters, first
published in 1953.]
\par
{\sc N. Ahmed and P. S. Fisher,} Study of algorithmic properties of
Chebyshev coefficients, {\em Int.\ J. Comp.\ Math.}\ 2 (1970), 307--317.
[Possibly the first paper to point out that Chebyshev coefficients
can be computed by Fast Fourier Transform.]
\par
{\sc A. C. Aitken,} On Bernoulli's numerical solution of
algebraic equations, {\em Proc.\ Roy.\ Soc.\ Edinb.}\ 46
(1926), 289--305.
\par
{\sc B. K. Alpert and V. Rokhlin,} A fast algorithm for the evaluation
of Legendre expansions, {\em SIAM J. Sci.\ Stat.\ Comp.}\ 12
(1991), 158--179. [Algorithm for converting between Legendre and Chebyshev
expansion coefficients.]
\par
{\sc A. Amiraslani,}
{\em New Algorithms for Matrices, Polynomials, and Matrix Polynomials,}
PhD diss., U. Western Ontario, 2006.
[Algorithms related to rootfinding by values rather than Cheybyshev coefficients.]
\par
{\sc A. Amiraslani, R. M. Corless, L. Gonzalez-Vega and A. Shakoori,}
Polynomial algebra by values, TR-04-01,
Ontario Research Center for Computer Algebra, www.orcca.on.ca, 2004.
[Outlines eigenvalue-based algorithms for finding roots of polynomials
from their values at sample points
rather than from coefficients in an expansion.]
\par
{\sc A. C. Antoulas,} {\em Approximation of Large-Scale Dynamical Systems},
SIAM, 2005.
[Textbook about model reduction, a subject making much
use of rational approximation.]
\par
{\sc A. I. Aptekarev,} Sharp constants for rational approximations of
analytic functions, {\em Math.\ Sbornik} 193 (2002), 1--72.
[Extends the result of Gonchar \& Rakhmanov 1989 on rational
approximation of $e^x$ on $(-\infty,0\kern .5pt]$ to give the precise
asymptotic form $E_{nn}\sim 2H^{n+1/2}$ first conjectured
by Magnus, where $H$ is Halphen's constant.]
\par
{\sc T. Bagby and N. Levenberg,} Bernstein theorems, {\em New Zeal.\ J. Math.}\ 22
(1993), 1--20.
[Presentation of four proofs of Bernstein's result that best polynomial
approximants to a function $f\in C([-1,1])$ converge geometrically if
and only if $f$ is analytic, with discussion of extension to higher dimension.]
\par
{\sc G. A. Baker, Jr. and P. Graves-Morris,} {\em Pad\'e Approximants,}
2nd ed., Cambridge U. Press, 1996.  [The standard reference on many
aspects of Pad\'e approximations and their applications.]
\par
{\sc N. S. Bakhvalov,} On the optimal speed of integrating analytic functions,
{\em Comput.\ Math.\ Math.\ Phys.}\ 7 (1967), 63--75.
[A theoretical paper that introduces the idea
of going beyond polynomials to speed up
Gauss quadrature by means of a change of variables/conformal map,
as in Hale \& Trefethen 2008.]
\par
{\sc S. Barnett (1975a),} A companion matrix analogue for orthogonal polynomials,
{\em Lin.\ Alg.\ Applics.}\ 12 (1975), 197--208.
[Generalization of Good's colleague matrices to orthogonal
polynomials other than Chebyshev.  Barnett apparently did not know that
Specht 1957 had covered the same ground.]
\par
{\sc S. Barnett (1975b),} Some applications of the comrade matrix,
{\em Internat.\ J. Control\/} 21 (1975), 849--855.
[Further discussion of comrade matrices.]
\par
{\sc Z. Battles,} {\em Numerical Linear Algebra for Continuous Functions,}
DPhil thesis, Oxford U. Computing Laboratory, 2005.
[Presentation of Chebfun, including description of Chebfun's rootfinding algorithm
based on recursion and eigenvalues of colleague matrices.]
\par
{\sc Z. Battles and L. N. Trefethen,} An extension of Matlab to continuous
functions and operators, {\em SIAM J. Sci.\ Comp.}\ 25 (2004), 1743--1770.
[Chebfun was conceived on December 8, 2001, and this was the
first publication about it.]
\par
{\sc F. L. Bauer,} The quotient-difference and epsilon algorithms,
in R. E. Langer, ed., {\em On Numerical Approximation,} U. Wisconsin Press,
1959, pp.~361--370.  [Introduction of the eta extrapolation algorithm
for series.]
\par
{\sc R. Bellman, B. G. Kashef and J. Casti,} Differential quadrature:
a technique for the rapid solution of nonlinear partial differential
equations, {\em J. Comp.\ Phys.}\ 10 (1972), 40--52.  [Perhaps the
first publication to give the formula for entries of a spectral
differentiation matrix.]
\par 
\end{par} \vspace{1em}
\begin{par}
 \vspace{-1em} \small \parskip=2pt
\def\parr{{\tiny\sl ~CHECK}\par}
{\sc S. N. Bernstein,} Sur l'approximation des fonctions continues
par des polyn\^omes, {\em Compt.\ Rend.\ Acad.\ Sci.}\ 152 (1911), 502--504.
[Announcement of some results proved in Bernstein 1912b.]
\par
{\sc S. N. Bernstein (1912a),} Sur les recherches r\'ecentes relatives
\`a la meilleure approximation des fonctions
continues par des polyn\^omes, {\em Proc.\ 5th Intern.\ Math.\ Congress,
v. 1}, 1912, 256--266.  [Announcement of the results of Bernstein
and Jackson on polynomial approximation,
including a table summarizing theorems by Bernstein, Jackson and Lebesgue
linking smoothness to rate of convergence.]
\par
{\sc S. N. Bernstein (1912b),} {\em Sur l'ordre de la meilleure approximation des
fonctions continues par des polyn\^omes de degr\'e
donn\'e}, M\'em.\ Acad.\ Roy.\ Belg., 1912, pp.~1--104.
[Major work (which won a prize from the Belgian Academy of Sciences)
establishing a number of the Jackson and Bernstein theorems
on rate of convergence of best approximations for differentiable or
analytic $f$.  Bernstein's fundamental estimates for functions analytic
in an ellipse appear in Sections 9 and 61.]
\par
{\sc S. N. Bernstein (1912c),}
Sur la valeur asymptotique de la meilleure approximation des fonctions
analytiques, {\em Compt.\ Rend.\ Acad.\ Sci.}\ 155 (1912), 1062--1065.  [One of the
first appearances of Bernstein ellipses, used here to analyze convergence
of best approximations for a function with a single real singularity
on the ellipse.]
\par
{\sc S. N. Bernstein (1912d),}
D\'emonstration du th\'eor\`eme de Weierstrass fond\'ee sur
le calcul des probabilit\'es, {\em Proc.\ Math.\ Soc.\
Kharkov} 13 (1912), 1--2.
[Bernstein's proof of the Weierstrass approximation theorem
based on Bernstein polynomials.]
\par
{\sc S. N. Bernstein (1914a),}
Sur la meilleure approximation des fonctions analytiques poss\'edant
des singularit\'es complexes,
{\em Compt.\ Rend.\ Acad.\ Sci.}\ 158 (1914), 467--469.  [Generalization of Bernstein
1912c to functions with a conjugate pair of singularities.]
\par
{\sc S. N. Bernstein (1914b),}
Sur la meilleure approximation de $|x|$ par des polyn\^omes
de degr\'es donn\'es, {\em Acta Math.}\ 37
(1914), 1--57.  [Investigates polynomial best approximation of
$|x|$ on $[-1,1]$ and mentions as a ``curious coincidence''
that $n E_n \approx 1/2\sqrt{\pi}$, a value that became known as
the ``Bernstein conjecture,'' later shown false by Varga and Carpenter.]
\par
{\sc S. N. Bernstein}, Quelques remarques sur l'interpolation,
{\em Math.\ Annal.}\ 79 (1919), 1--12.
[Written in 1914 but delayed in publication by the war,
this paper, like Faber 1914, pointed out that no array of nodes
for interpolation could yield convergence for all continuous
functions.]
\par
{\sc S. N. Bernstein}, Sur la limitation des valeurs d'un
polyn\^omes $P(x)$ de degr\'e $n$ sur tout un segment par
ses valeurs en $(n+1)$ points du segment, {\em Izv.\ Akad.\ Nauk
SSSR} 7 (1931), 1025--1050.
[Discussion of the problem of optimal interpolation nodes, defined
by minimization of the Lebesgue constant.]
\par
{\sc S. N. Bernstein}, On the inverse problem of the theory of the
best approximation of continuous functions, {\em Sochineya} 2 (1938),
292--294.
[Bernstein's lethargy theorem.]
\par
{\sc J.-P. Berrut,}
Rational functions for guaranteed and experimentally
well-conditioned global interpolation, {\em Comput.\ Math.\ Appl.}\ 15
(1988), 1--16.
[Observes that if the barycentric formula is applied on an arbitrary
grid with weights $1,-1,1,-1,\dots$ or $\textstyle{1\over 2}, -1, 1,-1\dots,$
the resulting rational interpolants are pole-free and accurate.]
\par

\end{par} \vspace{1em}
\begin{par}
 \vspace{-1em} \small \parskip=2pt
\def\parr{{\tiny\sl ~CHECK}\par}
{\sc J.-P. Berrut, R. Baltensperger and H. D. Mittelmann,} Recent developments
in barycentric rational interpolation, {\em Intern.\ Ser.\ Numer.\ Math.}\ 151
(2005), 27--51.
[Combines conformal maps with the rational barycentric formula to get
high-accuracy approximations of difficult functions.]
\par
{\sc J.-P. Berrut, M. S. Floater and G. Klein,} Convergence rates of
derivatives of a family of barycentric rational interpolants,
{\em Appl.\ Numer.\ Math.}\ 61 (2011), 989--1000.
[Establishes convergence rates for derivatives of Floater--Hormann barycentric
rational interpolants.]
\par
{\sc J.-P. Berrut and L. N. Trefethen,} Barycentric Lagrange interpolation,
{\em SIAM Rev.}\ 46 (2004), 501--517.  [Review of barycentric formulas for
polynomial and trigonometric interpolation.]
\par
{\sc A. Birkisson and T. Driscoll,} Automatic Fr\'echet differentiation
for the numerical solution of boundary-value problems.
{\em ACM Trans.\ Math.\ Softw.,} to appear, 2012.
[Description of Chebfun's method for solving nonlinear differential
equations based on Newton or damped-Newton
iteration and Automatic Differentiation.]
\par
{\sc H.-P. Blatt, A. Iserles and E. B. Saff,} Remarks on the
behaviour of zeros of best approximating polynomials and
rational functions, in J. C. Mason and M. G. Cox,
{\em Algorithms for Approximation,} Clarendon Press, 1987,
pp.~437--445.
[Shows that the type $(n,n)$ best rational approximations
to $|x|$ on $[-1,1]$ have all their zeros and poles on the imaginary
axis and converge to $x$ in the right half-plane
and to $-x$ in the left half-plane.]
\par
{\sc H.-P. Blatt and E. B. Saff,} Behavior of zeros of polynomials of
near best approximation, {\em J. Approx.\ Th.}\ 46 (1986), 323--344.
[Shows that if $f\in C([-1,1])$ is not analytic on $[-1,1]$, then the
roots of its best approximants $\{p_n^*\}$ cluster at every
point of $[-1,1]$ as $n\to\infty$.]
\par
{\sc H. F. Blichfeldt,} Note on the functions of the form
$f(x) \equiv \phi(x) + a_1 x^{n-1} +
a_2 x^{n-2} + \cdots + a_n$ which in a given interval
differ the least possible from zero, {\em Trans.\ Amer.\ Math.\
Soc.}\ 2 (1901), 100--102.
[Blichfeldt proves a part of the equioscillation theorem:
optimality implies equioscillation.]
\par
{\sc M. B\^ocher,} Introduction to the theory of Fourier's
series, {\em Ann.\ Math.}\ 7 (1906), 81--152.  [The paper that named
the Gibbs phenomenon.]
\par 
\end{par} \vspace{1em}
\begin{par}
 \vspace{-1em} \small \parskip=2pt
\def\parr{{\tiny\sl ~CHECK}\par}
{\sc E. Borel,} {\em Le\c cons sur les fonctions de variables
r\'eelles et les d\'eveloppements en series de polyn\^omes,}
Gauthier-Villars, Paris, 1905.
[The first textbook essentially about approximation theory, including
a proof of the equioscillation theorem, which Borel attributes
to Kirchberger.]
\par
{\sc F. Bornemann, D. Laurie, S. Wagon and J. Waldvogel,}
{\em The SIAM 100-Digit Challenge: A Study in High-Accuracy
Numerical Computing,} SIAM, 2004.
[Detailed study of ten problems whose answers are each a
single number, nine of which the authors manage to compute to
10,000 digits of accuracy through the use of ingenious algorithms
and acceleration methods.]
\par
{\sc J. P. Boyd,} {\em Chebyshev and Fourier Spectral Methods,} 2nd ed., Dover, 2001.
[A 668-page treatement of the subject with a great deal of
practical information.]
\par
{\sc J. P. Boyd,} Computing zeros on a real interval through Chebyshev
expansion and polynomial rootfinding, {\em SIAM J. Numer.\ Anal.}\ 40
(2002), 1666--1682.  [Proposes recursive
Chebyshev expansions for finding roots of real functions,
the idea that is the basis of the {\tt roots} command in Chebfun.]
\par
{\sc D. Braess,} On the conjecture of Meinardus on rational approximation
to $e^x$. II, {\em J. Approx.\ Th.}\ 40 (1984), 375--379.
[Establishes an asymptotic formula conjectured by Meinardus
for the best approximation error of $e^x$ on $[-1,1]$.]
\parr
{\sc D. Braess,} {\em Nonlinear Approximation Theory,} Springer, 1986.
[Advanced text on rational approximation and other topics, with emphasis
on methods of functional analysis.]
\par
{\sc C. Brezinski,} Extrapolation algorithms and Pad\'e approximations:
a historical survey, {\em Appl.\ Numer.\ Math.}\ 20 (1996), 299--318.
[Historical survey.]
\par
{\sc C. Brezinski and M. Redivo Zaglia,} {\em Extrapolation Methods: Theory
and Practice,} North-Holland, 1991.  [Extensive survey.]
\par
{\sc L. Brutman,} On the Lebesgue function for polynomial
interpolation, {\em SIAM J. Numer.\ Anal.}\ 15 (1978), 694--704.
[Sharpening of a result of Erd\H os 1960
concerning Lebesgue constants.]
\par
{\sc L. Brutman,} Lebesgue functions for polynomial interpolation---a survey,
{\em Ann.\ Numer.\ Math.}\ 4 (1997), 111--127.  [Exceptionally useful survey, including
detailed results on interpolation in Chebyshev points.]
\par
{\sc P. Butzer and F. Jongmans,} P. L. Chebyshev
(1821--1894): A guide to his life and work,
{\em J. Approx.\ Th.}\ 96 (1999), 111--138.
[Discussion of the leading Russian mathematician of
the 19th century.]
\par
{\sc C. Canuto, M. Y. Hussaini, A. Quarteroni and T. A. Zang,}
{\em Spectral Methods: Fundamentals in Single Domains},
Springer, 2006.  [A major monograph on both collocation
and Galerkin spectral methods.]
\par
{\sc C. Carath\'eodory and L. Fej\'er,} \"Uber den
Zusammenhang der Extremen von harmonischen Funktionen mit ihrer
Koeffizienten und \"uber den Picard-Landauschen Satz,
{\em Rend.\ Circ.\ Mat.\ Palermo} 32 (1911), 218--239.
[The paper that led, together with Schur 1918, to the
connection of approximation problems with eigenvalues and singular
values of Hankel
matrices, later the basis of the Carath\'eodory--Fej\'er method for
near-best approximation.]
\par
{\sc A. J. Carpenter, A. Ruttan and R. S. Varga,}
Extended numerical computations on the ``1/9'' conjecture in
rational approximation theory, in P. Graves-Morris, E. B. Saff and
R. S. Varga, eds., {\em Rational Approximation and Interpolation,}
Lect.\ Notes Math.\ 1105, Springer, 1984.
[Calculation to 40 significant digits of the best rational approximations
to $e^x$ on $(-\infty,0\kern .5pt]$ of types $(0,0),(1,1),\dots,(30,30)$.]
\par
{\sc A. L. Cauchy,} Sur la formule de Lagrange relative \`a
l'interpolation, {\em Cours d'Analyse de l'\'Ecole Royale Polytechnique: Analyse alg\'ebrique,}
Imprimerie Royale, Paris, 1821.
[First treatment of the ``Cauchy interpolation problem'' of interpolation by
rational functions.]
\par
{\sc A. L. Cauchy,} Sur un nouveau genre de calcul analogue au calcul infinit\'esimal,
{\em Exerc.\ Math\'ematiques} 1 (1926), 11--24.
[One of Cauchy's foundational texts on residue calculus, including
a derivation of what became known as the Hermite integral formula.]
\par
{\sc P. L. Chebyshev,} Th\'eorie des m\'ecanismes connus sous le nom
de parall\'elogrammes, {\em M\'em.}\ {\em Acad.\ Sci.\ P\'etersb.,} Series
7 (1854), 539--568.
[Introduction of the idea of best approximation by polynomials in the supremum norm.]
\par
{\sc P. L. Chebyshev,} Sur les questions de minima qui se rattachent
\`a la repr\'esentation approximative des fonctions,
{\em M\'em.\ Acad.\ Sci. P\'etersb.}\ Series 7 (1859), 199--291.
[Chebyshev's principal work on best approximation.]
\par 
\end{par} \vspace{1em}
\begin{par}
 \vspace{-1em} \small \parskip=2pt
\def\parr{{\tiny\sl ~CHECK}\par}
{\sc E. W. Cheney,} {\em Introduction to Approximation Theory,}
Chelsea, 1999.
[Classic approximation theory text first published in 1966.]
\par
{\sc J. F. Claerbout,} {\em Imaging the Earth's Interior,}
Blackwell, 1985.
[Text about the mathematics of migration for earth imaging by the
man who developed many of these techniques, based
on rational approximations of pseudodifferential operators.]
\par
{\sc C. W. Clenshaw and A. R. Curtis,} A method for numerical
integration on an automatic computer, {\em Numer.\ Math.}\ 2 (1960), 197--205.
[Introduction of Clenshaw--Curtis quadrature.]
\par
{\sc C. W. Clenshaw and K. Lord,} Rational approximations from Chebyshev
series, in B. K. P. Scaife, ed., {\em Studies
in Numerical Analysis,} Academic Press, 1974, pp.~95--113.
\par
{\sc W. J. Cody,} The FUNPACK package of special function
subroutines, {\em ACM Trans.\ Math.\ Softw.}\ 1 (1975), 13--25.
[Codes for evaluating special functions based on
rational approximations.]
\par
{\sc W. J. Cody,} Algorithm 715: SPECFUN---A portable FORTRAN
package of special function routines and test drivers,
{\em ACM Trans.\ Math.\ Softw.}\ 19 (1993), 22--32.
[Descendant of FUNPACK with greater portability.]
\par
{\sc W. J. Cody, W. Fraser and J. F. Hart,} Rational Chebyshev
approximation using linear equations, {\em Numer.\ Math.}\ 12
(1968), 242--251.
[Algol 60 code for best rational approximation by a variant of the Remes algorithm.]
\par
{\sc W. J. Cody, G. Meinardus and R. S. Varga,} Chebyshev rational
approximations to $e^{-x}$ in $[\kern .5pt 0,+\infty)$ and applications to
heat-conduction problems, {\em J. Approx.\ Th.}\ 2 (1969), 50--65.
[Introduces the problem of approximation of $e^{-x}$ on $[\kern .5pt 0,\infty)$, or
equivalently $e^{x}$ on $(-\infty,0\kern .5pt ]$, and shows that rational
best approximants converge geometrically.]
\par
{\sc R. M. Corless and S. M. Watt,} Bernstein bases are optimal, but, sometimes,
Lagrange bases are better, 2004, Proc.\ SYNASC (Symbolic and Numeric
Algorithms for Scientific Computing), Timisoara, 2004, pp.~141--152.
[A contribution to polynomial rootfinding with a marvelous title.]
\par
{\sc G. Darboux,} M\'emoire sur l'approximation des fonctions de
tr\`es-grands nombres, et sur une classe \'etendue de d\'eveloppements en
s\'erie, {\em J. Math.\ Pures Appl.}\ 4 (1878), 5--56.
\parr
{\sc S. Darlington,} Analytical approximations to approximations in the
Chebyshev sense, {\em Bell System Tech.\ J.}\ 49 (1970), 1--32.
[A precursor to the Carath\'eodory--Fej\'er method.]
\par
{\sc P. J. Davis,} {\em Interpolation and Approximation,} Dover, 1975.
[A leading text on the subject, first published in 1963.]
\par
{\sc P. J. Davis and P. Rabinowitz,} {\em Methods of Numerical Integration,} 2nd ed.,
Academic Press, 1984.
[The leading reference on numerical integration, with detailed
information on many topics, first published in 1975.]
\par
{\sc D. M. Day and L. Romero,} Roots of polynomials expressed in terms
of orthogonal polynomials, {\em SIAM J. Numer.\ Anal.}\ 43 (2005), 1969--1987.
[A rediscovery of the results of Specht, Good, Barnett and others on
colleague and comrade matrices.]
\par
{\sc C. de Boor and A. Pinkus,} Proof of the conjectures of Bernstein
and Erd\H os concerning the optimal nodes for polynomial interpolation,
{\em J. Approx.\ Th.}\ 24 (1978), 289--303.
[Together with Kilgore 1978, one of the papers solving
the theoretical problem of optimal interpolation.]
\par
{\sc R. A. DeVore and G. G. Lorentz,} {\em Constructive Approximation,}
Springer, 1993.  [An monograph emphasizing advanced topics.]
\par
{\sc Z. Ditzian and V. Totik,} {\em Moduli of Smoothness},
Springer-Verlag, New York, 1987.
[Careful analysis of smoothness and its effect on polynomial
approximation on an interval, including the dependence on location in the interval.]
\par
{\sc T. A. Driscoll, F. Bornemann and L. N. Trefethen,}
The chebop system for automatic solution of differential equations,
{\em BIT Numer.\ Math.}\ 48 (2008), 701--723.
[Extension of Chebfun to solve differential and integral equations.]
\par
{\sc T. A. Driscoll and N. Hale,} Resampling methods for boundary conditions
in spectral collocation, paper in preparation, 2012.
[Introduction of spectral collocation methods based on
rectangular matrices.]
\par
{\sc M. Dupuy,} Le calcul num\'erique des fonctions par
l'interpolation barycentrique, {\em Compt.\ Rend.\ Acad.\ Sci.}\ 226 (1948), 158--159.
[This paper is apparently the first to use the expression ``barycentric
interpolation'' and also the first to discuss barycentric interpolation
for non-equidstant points, the situation considered by Taylor 1945.]
\par
{\sc A Dutt, M. Gu and V. Rokhlin,} Fast algorithms for polynomial
interpolation, integration, and differentiation, {\em SIAM J. Numer.\
Anal.}\ 33 (1996), 1689--1711.  [Uses the Fast Multipole Method to derive
fast algorithms for non-Chebyshev points.]
\par
{\sc H. Ehlich and K. Zeller,} Auswertung der Normen von
Interpolationsoperatoren, {\em Math.\ Ann.}\ 164 (1966), 105--112.
[Bound on Lebesgue constant for interpolation in Chebyshev points.]
\par
{\sc D. Elliott,} A direct method for ``almost'' best uniform approximation,
in {\em Error, Approximation, and Accuracy,} eds. F. de Hoog and C. Jarvis,
U. Queensland Press, St. Lucia, Queensland, 1973, 129--143.
[A precursor to the Carath\'eodory--Fej\'er method.]
\par
{\sc M. Embree and D. Sorensen,} {\em An Introduction to Model Reduction
for Linear and Nonlinear Differential Equations,} to appear.
[Textbook.]
\par
{\sc B. Engquist and A. Majda,} Absorbing boundary conditions
for the numerical simulation of waves, {\em Math.\ Comput.}\ 31
(1977), 629--651.
[Highly influential paper on the use of Pad\'e approximations to
a pseudodifferential operator to develop numerical boundary conditions.]
\par

\end{par} \vspace{1em}
\begin{par}
 \vspace{-1em} \small \parskip=2pt
\def\parr{{\tiny\sl ~CHECK}\par}
{\sc P. Erd\H os,} Problems and results on the theory of interpolation. II,
{\em Acta Math.\ Acad.\ Sci.\ Hungar.}\ 12 (1961), 235--244.
[Shows that Lebesgue constants for optimal interpolation points are no better than
for Chebyshev points asymptotically as $n\to\infty$.]
\par
{\sc T. O. Espelid,} Extended doubly adaptive quadrature routines,
Tech.\ Rep.\ 266, Dept.\ Informatics, U. Bergen, Feb.\ 2004.
[Presentation of {\tt coteda} and {\tt da2glob} quadrature codes.]
\par
{\sc L. Euler,} {\em De Seriebus Divergentibus,}
Novi Commentarii academiae scientiarum Petropolitanae 5, (1760) (205).
[Early work on divergent series.]
\par
{\sc L. Euler,} De eximio usu methodi interpolationum in serierum doctrina,
{\em Opuscula Analytica} 1 (1783), 157--210.
[A work on various applications of interpolation, including equations
related to the Newton and Lagrange formulas for polynomial interpolation.]
\par
{\sc L. C. Evans and R. F. Gariepy,} {\em Measure Theory and Fine Properties
of Functions}, CRC Press, 1991.
[Includes a definition of the
total variation in the measure theoretic context.]
\par
{\sc G. Faber,} \"Uber die interpolatorische Darstellung
stetiger Funktionen, {\em Jahresber. Deutsch.\ Math.\ Verein.}\ 23 (1914),
190--210.  [Shows that no fixed system of nodes for polynomial
interpolation will lead to convergence for all continuous $f$.]
\par
{\sc L. Fej\'er,} Sur les fonctions born\'ees et
int\'egrables, {\em Compt.\ Rend.\ Acad.\ Sci.}\ 131 (1900),
984--987.
[Fej\'er, age 20, provides a new method of summing divergent
Fourier series, with a new proof of the Weierstrass approximaton
theorem as a corollary.]
\parr
{\sc L. Fej\'er,} Lebesguesche Konstanten und divergente
Fourierreihen, {\em J. f.\ Math.}\ 138 (1910), 22--53.
[Shows that Lebesgue constants for Fourier projection are
asymptotic to $(4/\pi^2)\log n$ as $n\to\infty$.]
\par
{\sc L. Fej\'er,} Ueber Interpolation, {\em Nachr.\ Gesell.\ Wiss.\ G\"ottingen
Math.\ Phys.\ Kl.}\ (1916), 66--91.
[Proves the Weierstrass approximation theorem by showing that
Hermite--Fej\'er interpolants in Chebyshev points of the
first kind converge for any $f\in C([-1,1])$.]
\par
{\sc A. M. Finkelshtein,} Equilibrium problems of potential theory
in the complex plane, in {\em Orthogonal Polynomials and Special Functions},
Lect.\ Notes Math.\ 1883, pp.~79--117, Springer, 2006.
[Survey article.]
\par
{\sc M. S. Floater and K. Hormann,} Barycentric rational interpolation
with no poles and high rates of approximation,
{\em Numer.\ Math.}\ 107 (2007), 315--331.
[Extension of results of Berrut 1988 to
a family of barycentric rational interpolants of arbitrary order.]
\par
{\sc G. B. Folland,} {\em Introduction to Partial Differential Equations},
2nd ed., Princeton U. Press, 1995.  [An elegant introduction
to PDEs published first in 1976, including the Weierstrass approximation theorem proved
via the heat equation and generalized to multiple dimensions.]
\par
{\sc B. Fornberg,} Generation of finite difference formulas
on arbitrarily spaced grids, {\em Math.\ Comp.}\ 51 (1988),
699--706. [Stable algorithm for generating finite difference
formulas on arbitrary grids.]
\par
{\sc B. Fornberg,} {\em A Practical Guide to Pseudospectral Methods,}
Cambridge U. Press, 1996.
[Practically-oriented textbook of spectral collocation methods for
solving ordinary and partial differential equations, based on
Chebyshev interpolants.]
\par
{\sc S. Fortune,} Polynomial root finding using iterated
eigenvalue computation,
{\em Proc.\ 2001 Intl.\ Symp.\ Symb.\ Alg.\ Comput.}, ACM, 2001,
pp.~121--128. [An eigenvalue-based rootfinding algorithm
that works directly from data samples rather than expansion
coefficients.]
\par
{\sc L. Fox and I. B. Parker,} {\em Chebyshev Polynomials in
Numerical Analysis,} Oxford U. Press, 1968.  [A precursor to
the work of the 1970s and later on Chebyshev spectral methods.]
\par
{\sc J. G. F. Francis,} The QR transformation: a unitary analogue
to the LR transformation, parts I and II, {\em Computer J.}
4 (1961), 256--272 and 332--345.  [Introduction of the QR
algorithm for numerical computation of matrix eigenvalues.]
\par
{\sc G. Frobenius,} Ueber Relationen zwischen den N\"aherungsbr\"uchen von
Potenzreihen, {\em J. Reine Angew.\ Math.}\ 90 (1881), 1--17.
[The first systematic treatment of Pad\'e approximation.]
\par
{\sc M. Froissart,} Approximation de Pad\'e: application \`a la physique des
particules \'el\'ementaires, {\em RCP, Programme No.~25}, v.~9,
CNRS, Strasbourg (1969), pp.~1--13.
[A rare publication by the mathematician and physicist after whom
Froissart doublets were named (by Bessis).]
\par
{\sc D. Gaier,} {\em Lectures on Complex Approximation},
Birkh\"auser, 1987.
[A shorter book presenting some of the material
considered at greater length in Smirnov \& Lebedev 1968 and Walsh 1969.]
\par
{\sc C. F. Gauss,} Methodus nova integralium valores per approximationem
inveniendi, {\em Comment.\ Soc.\ Reg.\ Scient.\ Gotting.\ Recent.}, 1814,
pp.~39--76.
[Introduction of Gauss quadrature---via continued fractions, not
orthogonal polynomials.]
\par
{\sc W. Gautschi,} A survey of Gauss--Christoffel quadrature
formulae, in P. L. Butzer and F. Feh\'er, eds.,
{\em E. B. Christoffel: The Influence of His Work in Mathematics
and the Physical Sciences,} Birkh\"auser, 1981, pp.~72--147.
[Outstanding survey of many aspects of Gauss quadrature.]
\par
{\sc W. Gautschi,} {\em Orthogonal Polynomials: Computation and Approximation},
Oxford U. Press, 2004.  [A monograph on orthogonal polynomials with emphasis on
numerical aspects.]
\par
{\sc K. O. Geddes,} Near-minimax polynomial approximation in
an elliptical region, {\em SIAM J. Numer.\ Anal.}\ 15 (1978), 1225--1233.
[Chebyshev expansions via FFT for analytic functions on an interval.]
\par
{\sc W. M. Gentleman (1972a),} Implementing Clenshaw--Curtis quadrature,
I: Methodology and experience, {\em Comm.\ ACM} 15 (1972), 337--342.
[A surprisingly modern paper that includes the aliasing formula
for Chebyshev polynomials.]
\par
{\sc W. M. Gentleman (1972b),} Implementing Clenshaw--Curtis quadrature,
II: Computing the cosine transformation, {\em Comm.\ ACM} 15 (1972), 343--346.
[First connection of Clenshaw--Curtis quadrature with FFT.]
\par
{\sc A. Glaser, X. Liu and V. Rokhlin,} A fast algorithm for the
calculation of the roots of special functions, {\em SIAM J. Sci.\ Comp.}\ 29
(2007), 1420--1438.  [Introduction of an algorithm for computation
of Gauss quadrature nodes and weights in $O(n)$ operations rather
than $O(n^2)$ as in Golub \& Welsch 1969.]
\par
{\sc K. Glover,} All optimal Hankel-norm approximations of linear
multivariable systems and their $L^\infty$-error bounds,
{\em Internat.\ J. Control} 39 (1984), 1115--1193.
[Highly influential article on rational approximations
in control theory.]
\par
{\sc S. Goedecker,} Remark on algorithms to find roots
of polynomials, {\em SIAM J. Sci.\ Comput.}\ 15 (1994), 1059--1063.
[Emphasizes the stability of companion matrix eigenvalues as
an algorithm for polynomial rootfinding, given a polynomial
expressed by its coefficients in the monomial basis.]
\par

\end{par} \vspace{1em}
\begin{par}
 \vspace{-1em} \small \parskip=2pt
\def\parr{{\tiny\sl ~CHECK}\par}
{\sc G. H. Golub and J. H. Welsch,} Calculation of Gauss
quadrature rules, {\em Math.\ Comp.}\ 23 (1969), 221--230.
[Presentation of the famous $O(n^2)$ algorithm for Gauss quadrature
nodes and weights via a tridiagonal Jacobi matrix eigenvalue problem.]
\par
{\sc A. A. Gonchar and E. A. Rakhmanov,} Equilibrium
distributions and degree of rational approximation
of analytic functions, {\em Math.\ USSR Sbornik} 62 (1989), 305--348.
[A landmark paper, first published in Russian in 1987, that
applies methods of potential theory to prove
that the optimal rate of convergence for type $(n,n)$ rational
minimax approximations of $e^x$ on $(-\infty,0\kern .5pt]$ is
$O((9.28903\dots)^{-n})$ as $n\to\infty$.]
\par
{\sc V. L. Goncharov,} The theory of best approximation of functions,
{\em J. Approx.\ Th.}\ 106 (2000), 2--57.
[English translation of a 1945 historical survey in Russian emphasizing
contributions of Chebyshev and his successors.]
\par
{\sc P. Gonnet, S. G\"uttel and L. N. Trefethen,} Robust Pad\'e approximation
via SVD, {\em SIAM Rev.}, to appear.
[Introduction of the robust SVD-based algorithm for computing
Pad\'e approximants presented in Chapter~27.]
\par
{\sc P. Gonnet, R. Pach\'on and L. N. Trefethen,} Robust rational
interpolation and least-squares, {\em Elect.\ Trans.\ Numer.\ Anal.}\ 38
(2011), 146--167.
[A robust algorithm based on the singular value decomposition for computing
rational approximants without spurious poles.]
\par
{\sc I. J. Good,} The colleague matrix, a Chebyshev analogue of the
companion matrix, {\em Quart. J. Math.}\ 12 (1961), 61--68.
[Together with Specht 1960, one of the two original independent discoveries
that roots of polynomials in Chebyshev form can be computed as eigenvalues
of colleague matrices, a term introduced here.
Good recommends this approach to numerical rootfinding for functions other
than polynomials too.]
\par
{\sc D. Gottlieb, M. Y. Hussaini and S. A. Orszag,} Introduction:
theory and applications of spectral methods, in R. G. Voigt, D. Gottlieb and
M. Y. Hussaini, {\em Spectral Methods for Partial Differential
Equations}, SIAM, 1984.  [Early survey article on spectral collocation
methods, including the first publication of the formula for the
entries of Chebyshev differentiation matrices.]
\par
{\sc W. B. Gragg,} The Pad\'e table and its relation to certain
algorithms of numerical analysis, {\em SIAM Rev.}\ 14 (1972), 1--62.
[A careful and extensive mathematical reference on the structure and
algebra of the Pad\'e table as presented in Chapter 27, though
with an emphasis on determinants.]
\par
{\sc A. Greenbaum and L. N. Trefethen}, GMRES/CR and
Arnoldi/Lanczos as matrix approximation problems,
{\em SIAM J. Sci.\ Comput.}\ 15 (1994), 359--368.
[Shows that the GMRES/CR and Arnoldi/Lanczos matrix iterations are
equivalent to certain polynomial approximation problems and generalizes
this observation to matrix approximation problems such as ``ideal GMRES''.]
\par
{\sc T. H. Gronwall,} \"Uber die Gibbssche Erscheinung und die trigonometrischen
Summen $\sin x + {1\over 2} \sin 2x + \cdots + {1\over n} \sin nx$,
{\em Math.\ Ann.}\ 72 (1912), 228--243.
[Investigates detailed behavior of Fourier approximations
near Gibbs discontinuities.]
\par
{\sc M. H. Gutknecht,} Algebraically solvable Chebyshev approximation
problems, in C. K. Chui, L. L. Schumaker and J. D. Ward., eds.,
{\em Approximation Theory IV},
Academic Press, 1983.  [Shows that many examples of $\infty$-norm best approximations
that can be written down explicitly correspond to Carath\'eodory--Fej\'er
approximations.]
\par
{\sc M. H. Gutknecht,} In what sense is the rational interpolation problem
well posed?, {\em Constr.\ Approx.}\ 6 (1990), 437--450.
[Generalization of Trefethen \& Gutknecht 1985 from Pad\'e
to multipoint Pad\'e approximation.]
\par
{\sc M. H. Gutknecht and L. N. Trefethen,} Real polynomial
Chebyshev approximation by the Carath\'eodory--Fej\'er method,
{\em SIAM J. Numer.\ Anal.}\ 19 (1982), 358--371.
[Introduction of CF approximation on an interval.]
\par
{\sc S. G\"uttel,} {\em Rational Krylov Methods for Operator Functions},
PhD dissertation, TU Bergakademie Freiberg, 2010.  [Survey and
analysis of advanced methods of numerical linear algebra based
on rational approximations.]
\par
{\sc N. Hale, N. J. Higham and L. N. Trefethen,} Computing $A^\alpha$,
$\log(A)$, and related matrix functions by contour
integrals, {\em SIAM J. Numer.\ Math.}\ 46 (2008), 2505--2523.
[Derives efficient algorithms for computing matrix functions from
trapezoid rule approximations to contour integrals accelerated by
contour maps.  These are equivalent to rational approximations.]
\par
{\sc N. Hale and T. W. Tee,} Conformal maps to multiply slit
domains and applications, {\em SIAM J. Sci.\ Comput.}\ 31 (2009),
3195--3215.
[Extension of Tee \& Trefethen 2006 to new geometries and applications.]
\par
{\sc N. Hale and A. Townsend,}
Fast and accurate computation of Gauss--Jacobi quadrature nodes and weights,
manuscript in preparation, 2012.
[Proposes an $O(n)$ algorithm based on asymptotic formulas
for computing Gauss quadrature nodes and weights for large $n$,
much faster than the Glaser--Liu--Rokhlin
algorithm in a Matlab implementation.]
\par
{\sc N. Hale and L. N. Trefethen,}
New quadrature formulas from conformal maps,
{\em SIAM J. Numer.\ Anal.}\ 46 (2008), 930--948.  [Shows that conformal mapping
can be used to derive quadrature formulas that converge faster
than Gauss, as in Bakhvalov 1967.]
\par
{\sc N. Hale and L. N. Trefethen,}
Chebfun and numerical quadrature,
{\em Science in China,} to appear, 2012.  [Review of quadrature
algorithms in Chebfun, including fast Gauss and Gauss--Legendre
quadrature by the Glaser--Liu--Rokhlin algorithm (but not yet the
Hale--Townsend algorithm) with applications
to computing with functions with singularities.]
\par
{\sc L. Halpern and L. N. Trefethen,} Wide-angle one-way wave equations,
{\em J. Acoust.\ Soc.\ Amer.}\ 84 (1988), 1397--1404.
[Review of rational approximations to $\sqrt{1-s^2}$ on $[-1,1]$
for application to one-way wave equations.]
\par
{\sc G. H. Halphen,} Trait\'e des fonctions elliptiques et de
leurs applications, Gauthier-Villars, Paris, 1886.
[A treatise on elliptic functions that contains a calculation
to six digits of the number ${\approx\kern 1pt}1/9.28903$ that
later became known as ``Halphen's constant''
in connection with the rational approximation
of $e^x$ on $(-\infty,0\kern .5pt]$.]
\par
{\sc P. C. Hansen,} {\em Rank-Deficient and Discrete Ill-Posed Problems:
Numerical Aspects of Linear Inversion,} SIAM, 1998.
[A leading monograph on the treatment of rank-deficient or
ill-posed matrix problems.]
\par
{\sc G. H. Hardy,} {\em Divergent Series,}, revised ed., \'Editions Jacques Gabay, 1991.
[Hardy's marvelous posthumous volume on the mathematics of divergent
series, first published in 1949.]
\par
{\sc J. F. Hart et al.,} {\em Computer Approximations}, Wiley, 1968.
[A classic compendium on computer evaluation of special functions,
containing 150 pages of explicit coefficients of rational approximations.]
\par
{\sc E. Hayashi, L. N. Trefethen and M. H. Gutknecht,} The CF table,
{\em Constr. Approx.}\ 6 (1990), 195--223.  [The most systematic
and detailed treatment of the problem of rational CF approximation
of a function $f$ on the unit disk, including cases where $f$ is
just in the Wiener class or continuous on the unit circle.]
\par
{\sc G. Heinig and K. Rost,} {\em Algebraic Methods for
Toeplitz-like Matrices and Operators,} Birkh\"auser, 1984.
[Analyzes rank properties of Toeplitz and Hankel matrices related to the
robust Pad\'e algorithms of Chapter~27.]
\par
{\sc G. Helmberg and P. Wagner,} Manipulating Gibbs' phenomenon for
Fourier interpolation, {\em J. Approx.\ Th.}\ 89 (1997), 308--320.
[Analyzes the overshoot in various versions of the Gibbs phenomenon
for trigonometric interpolation.]
\par

\end{par} \vspace{1em}
\begin{par}
 \vspace{-1em} \small \parskip=2pt
\def\parr{{\tiny\sl ~CHECK}\par}
{\sc P. Henrici,} {\em Applied and Computational Complex Analysis, vols.\ 1--3},
Wiley, 1974 and 1977 and 1986.
[An extensive and highly readable account of
applied complex analysis, full of details that are hard to find elsewhere.]
\par
{\sc C. Hermite,} Sur la formule d'interpolation de Lagrange, {\em J. Reine
Angew.\ Math.}\ 84 (1878), 70--79.  [Application of what became known
as the Hermite integral formula for polynomial interpolation, which
had earlier been given by Cauchy, to problems of interpolation with
confluent data points.]
\par
{\sc J. S. Hestaven, S. Gottlieb and D. Gottlieb,} {\em Spectral Methods
for Time-Dependent Problems,} Cambridge U. Press, 2007.  [Well-known textbook
on spectral methods.]
\par
{\sc E. Hewitt and R. E. Hewitt,} The Gibbs--Wilbraham phenomenon: an episode
in Fourier analysis, {\em Arch.\ Hist.\ Exact Sci.}\ 21 (1979), 129--160.
[Discussion of the complex and not always pretty history of
attempts to analyze the Gibbs phenomenon.]
\par
{\sc N. J. Higham,} The numerical stability of barycentric Lagrange
interpolation, {\em IMA J. Numer.\ Anal.}\ 24 (2004), 547--556.
[Proves that barycentric interpolation in Chebyshev points is
numerically stable, following earlier work of Rack \& Reimer 1982.]
\par
{\sc N. J. Higham,} {\em Functions of Matrices: Theory and Computation,} SIAM, 2008.
[The definitive treatment of the problem of computing functions of
matrices as of 2008.  Many of the algorithms have connections with
polynomial or rational approximation.]
\par
{\sc N. J. Higham,} The scaling and squaring method for the
matrix exponential revisited,
{\em SIAM Rev.}\ 51 (2009), 747--764.
[Careful analysis of Matlab's method of evaluating
$e^A$ leads to several improvements in the algorithm and the
recommendation to use the Pad\'e approximant of type $(13,13)$.]
\par
{\sc N. J. Higham and A. H. Al-Mohy,} Computing matrix functions,
{\em Acta Numer.}\ 19 (2010), 159--208.  [Survey includes
an appendix comparing Pad\'e and Taylor approximants
for computing the exponential of a matrix.]
\par
{\sc E. Hille,} {\em Analytic Function Theory}, 2 vols., 2nd ed., Chelsea, 1973.
[Major work first published in 1959 and 1962.]
\par
{\sc M. Hochbruck and A. Ostermann,} Exponential integrators,
{\em Acta Numer.}\ 19 (2010), 209--286.
[Survey of exponential integrators for the fast numerical
solution of stiff ODEs and PDEs.]
\par
{\sc G. Hornecker}, D\'etermination des meilleures approximations
rationnelles (au sens de Tchebychef) des functions
r\'eelles d'une variable sur un segment fini et des bornes
d'erreur correspondantes, {\em Compt.\ Rend.\ Acad.\ Sci.}\ 249 (1956), 2265--2267.
[Possibly the first proposal of a kind
of Chebyshev--Pad\'e approximation for intervals.]
\par
{\sc J. P. Imhof,} On the method for numerical integration of
Clenshaw and Curtis, {\em Numer.\ Math.}\ 5 (1963), 138--141.
[Shows that the Clenshaw--Curtis quadrature weights are positive.]
\parr
{\sc A. Iserles,} A fast and simple algorithm for the computation of Legendre
coefficients, {\em Numer.\ Math.}\ 117 (2011), 529--553.
[A fast algorithm based on a numerical contour integral over an
ellipse in the complex plane.]
\par
{\sc D. Jackson,}
{\em \"Uber die Genauigkeit der Ann\"aherung
stetiger Funktionen durch ganze rationale Funktionen gegebenen Grades
und trigonometrische Summen gegebener Ordnung,} dissertation,
G\"ottingen, 1911.  [Jackson's PhD thesis under Landau in
G\"ottingen, which together with Bernstein's work at the same time
{\sc (1912b)} established many of the fundamental results of approximation theory.  Despite
the German, Jackson was an American from Massachusetts, like me---Harvard
Class of 1908.]
\par
{\sc D. Jackson,} On the accuracy of trigonometric interpolation,
{\em Trans.\ Amer.\ Math.\ Soc.}\ 14 (1913), 453--461.
[In the final paragraph of this paper, polynomial interpolation in
Chebyshev points (2.2) is considered, possibly for the first time in
the literature.]
\parr

\end{par} \vspace{1em}
\begin{par}
 \vspace{-1em} \small \parskip=2pt
\def\parr{{\tiny\sl ~CHECK}\par}
{\sc C. G. J. Jacobi,} {\em Disquisitiones Analyticae de Fractionibus
Simplicibus}, dissertation, Berlin, 1825.
[In his discussion of partial fractions Jacobi effectively states
the ``first form'' of the barycentric interpolation formula.]
\par
{\sc C. G. J. Jacobi,} \"Uber Gauss' neue Methode, die Werthe der
Integrale n\"aherungsweise zu finden, {\em J. Reine Angew.\ Math.}\ 1
(1826), 301--308.
[This paper first invents the subject of orthogonal
polynomials, then shows that Gauss quadrature can be
derived in this framework.]
\par
{\sc C. G. J. Jacobi,} \"Uber die Darstellung einer Reihe gegebener Werthe durch
eine gebrochene rationale Function,
{\em J. Reine Angew.\ Math.}\ 30 (1846), 127--156.
[Jacobi's major work on rational interpolation.]
\par
{\sc R. Jentzsch,} {\em Untersuchungen zur Theorie analytischer Funktionen,}
dissertation, Berlin, 1914. [Jentzsch, who was also a noted poet and was
killed at age 27 in World War I,
proves here that every point on the circle of convergence of a power series
is the limit of zeros of its partial sums.]
\par
{\sc D. C. Joyce,} Survey of extrapolation processes in numerical analysis,
{\em SIAM Rev.}\ 13 (1971), 435--490.
[Scholarly review of a wide range of material.]
\par
{\sc A.-K. Kassam and L. N. Trefethen,}
Fourth-order time-stepping for stiff PDEs, {\em SIAM J. Sci.\ Comput.}\ 26
(2005), 1214--1233.
[Application of exponential integrator formulas to efficient numerical
solution of stiff PDEs.]
\par
{\sc T. A. Kilgore,} A characterization of the Lagrange interpolating
projection with minimal Tchebycheff norm, {\em J. Approx.\ Th.}\ 24 (1978),
273--288.
[Together with de Boor \& Pinkus 1978, one of the papers solving
the theoretical problem of optimal interpolation.]
\par
{\sc P. Kirchberger,} {\em Ueber Tchebychefsche Ann\"aherungsmethoden,}
PhD thesis, G\"ottingen, 1902.
[Kirchberger's PhD thesis under Hilbert contains apparently the
first full statement and proof of the equioscillation theorem.]
\par
{\sc P. Kirchberger,} \"Uber Tchebychefsche Ann\"aherungsmethoden,
{\em Math.\ Ann.}\ 57 (1903), 509--540.
[Extract from his PhD thesis the year before, focusing on
multivariable extensions but without the equioscillation theorem.]
\par
{\sc A. N. Kolmogorov,} A remark on the polynomials of P. L. Chebyshev
deviating the least from a given function,
{\em Uspehi Mat.\ Nauk} 3 (1948), 216--221 [Russian].  [Criterion
for best complex approximations.]
\par
{\sc D. Kosloff and H. Tal-Ezer,} A modified Chebyshev pseudospectral
method with an $O(N^{-1})$ time step restriction,
{\em J. Comp.\ Phys.}\ 104 (1993), 457--469.
[Introduces a change of variables as a basis
for non-polynomial spectral methods.]
\par
{\sc A. B. J. Kuijlaars,} Convergence analysis of Krylov subspace
iterations with methods from potential theory,
{\em SIAM Rev.}\ 48 (2006), 3--40.
[Analyzes the connection between potential theory and the
roots of polynomial approximants implicitly constructed by
Krylov iterations such as the conjugate gradient and Lanczos iterations.]
\par
{\sc J. L. Lagrange,} Le\c cons \'el\'ementaires sur les
Math\'ematiques,  Le\c con V., {\em J. de l'\'Ecole polytechnique,}
Tome II, Cahier 8, pp.~274--278, Paris, 1795.  [Contains
what became known as the Lagrange interpolation formula,
published earlier by Waring 1779 and Euler 1783.]
\par
{\sc B. Lam,} {\em Some Exact and Asymptotic Results for Best
Uniform Approximation}, PhD thesis, U. of Tasmania, 1972.
[A precursor to the Carath\'eodory--Fej\'er method.]
\par
{\sc E. Landau,} Absch\"atzung der Koeffizientensumme einer Potenzreihe,
{\em Archiv Math.\ Phys.}\ 21 (1913), 42--50 and 250--255. [Investigates the
norm of the degree $n$ Taylor projection for functions analytic in the
unit disk, now known as the Landau constant,
showing it is asymptotic to $\pi^{-1} \log n$ as $n\to\infty$.]
\par
{\sc H. Lebesgue,} Sur l'approximation des fonctions, {\em Bull.\ Sci.\
Math.}\ 22 (1898), 278--287.  [In Lebesgue's first published paper,
he proves the Weierstrass approximation theorem by approximating
$|x|$ by polynomials and noting that any continuous function can
be approximated by piecewise linear functions.]
\par
{\sc A. L. Levin and E. B. Saff,} Potential theoretic tools in
polynomial and rational approximation, in J.-D. Fournier,
et al., eds., {\em Harmonic
Analysis and Rational Approximation,} Lec.\ Notes Control Inf.\ Sci.\
326/2006 (2006), 71--94.
[Survey article by two of the experts.]
\par
{\sc R.-C. Li,} Near optimality of Chebyshev interpolation for
elementary function computations, {\em IEEE Trans.\ Computers}
53 (2004), 678--687.
[Shows that although Lebesgue constants for Chebyshev points
grow logarithmically as $n\to\infty$, for many classes of functions
of interest the interpolants come within a factor of 2 of optimality.]
\par
{\sc E. L. Lindman,} `Free-space' boundary conditions for
the time-dependent wave equation,
{\em J. Comput.\ Phys.}\ 18 (1975), 66--78.
[Absorbing boundary conditions based on
Pad\'e approximation of a square root function, later developed
further by Engquist \& Majda 1977].
\par
{\sc G. G. Lorentz,} {\em Approximation of Functions}, 2nd ed., Chelsea, 1986.
[A readable treatment including good summaries of the Jackson theorems
for polynomial and trigonometric approximation, first published in 1966.]
\par
{\sc K. N. Lungu,} Best approximations by rational functions,
{\em Math.\ Notes} 10 (1971), 431--433.
[Shows that the best rational approximations to a real function
on an interval may be complex and hence also nonunique,
with examples as simple as
type $(1,1)$ approximation of $|x|$ on $[-1,1]$.]
\par
{\sc H. J. Maehly and Ch.\ Witzgall,} Tschebyscheff-Approximationen in kleinen
Intervallen.\ II.\ Stetikeitss\"atze f\"ur gebrochen rationale Approximationen,
{\em Numer.\ Math.}\ 2 (1960), 293--307.
[Investigates well-posedness of the Cauchy interpolation problem and asymptotics
of best approximations on small intervals.]
\par
{\sc A. P. Magnus,} CFGT determination of Varga's constant
'1/9', unpublished manuscript, 1985.
[First identification of the the exact value of Halphen's constant
$C=9.28903\dots$ for
the optimal rate of convergence $O(C^{-n})$ of best type $(n,n)$ approximations
to $e^x$ on $(-\infty,0\kern .5pt]$, later proved correct by Gonchar \& Rakhmanov 1989.]
\par
{\sc A. P. Magnus and J. Meinguet,} The elliptic functions and integrals
of the ``1/9'' problem, {\em Numer.\ Alg.}\ 24 (2000), 117--139.
[Summary of work initiated by Magnus relating potential theory, elliptic
functions, and the ``1/9'' problem.]
\par
{\sc J. Marcinkiewicz,} Quelques remarques sur l'interpolation,
{\em Acta Sci.\ Math.\ (Szeged)} 8 (1936--37), 127--30.
[In contrast to the result of Faber 1914, shows that for any fixed
continuous function $f$ there is an array of interpolation
nodes that leads to convergence as $n\to\infty$.]
\par
{\sc A. I. Markushevich,} {\em Theory of Functions of a Complex Variable},
2nd ed., 3 vols., Chelsea, 1985.  [A highly readable treatise
on complex variables first published in 1965,
including chapters on Laurent series, polynomial
interpolation, harmonic functions, and rational approximation.]
\par
{\sc J. C. Mason and D. C. Handscomb,} {\em Chebyshev Polynomials,}
Chapman and Hall/CRC, 2003.
[An extensive treatment of four varieties of
Chebyshev polynomials and their applications.]
\par
{\sc G. Mastroianni and M. G. Russo,} Some new results on
Lagrange interpolation for bounded variation functions,
{\em J. Approx.\ Th.}\ 162 (2010), 1417--1428.
[A collection of bounds in $L^p$ norms for both
$p<\infty$ and $p=\infty$.]
\par
{\sc G. Mastroianni and J. Szabados,} Jackson order of approximation
by Lagrange interpolation.\ II, {\em Acta Math.\ Acad.\ Sci.\ Hungar.}\ 69 (1995), 73--82.
[Corollary 2 bounds the rate of convergence of Chebyshev interpolants
for functions whose $k\kern -2pt$ th derivative has bounded variation.]
\par
{\sc J. H. McCabe and G. M. Phillips,} On a certain class of Lebesgue
constants, {\em BIT} 13 (1973), 434--442.  [Shows that the Lebesgue constant
for polynomial interpolation in $n+1$ Chebyshev points of the second
kind is bounded by that of $n$ Chebyshev points of the first kind.  The same
result had been found earlier by Ehlich \& Zeller 1966.]
\par
{\sc J. H. McClellan and T. W. Parks,} A personal history of the
Parks--McClellan algorithm, {\em IEEE Sign.\ Proc.\ Mag.}\ 82 (2005),
82--86.  [The story of the development of the celebrated
filter design algorithm published in Parks \& McClellan 1972.]
\par
{\sc G. Meinardus,} {\em Approximation of Functions: Theory and Numerical Methods,}
Springer, 1967.  [Classic approximation theory monograph.]
\par
{\sc C. M\'eray,} Observations sur la l\'egitimit\'e
de l'interpolation, {\em Annal.\ Scient.\ de l'\'Ecole Normale Sup\'erieure}
3 (1884), 165--176.  [Discussion of the
possibility of nonconvergence of polynomial interpolants 17 years before
Runge, though without so striking an example or conclusion.
M\'eray uses just the right technique, the Hermite integral formula, which he correctly
attributes to Cauchy.]
\par
{\sc C. M\'eray,} Nouveaux exemples d'interpolations illusoires,
{\em Bull.\ Sci.\ Math.}\ 20 (1896), 266--270.
[Continuation of M\'eray 1884 with more examples.]
\par
{\sc S. N. Mergelyan,} On the representation of functions by
series of polynomials on closed sets (Russian).  {\em Dokl.\ Adak.\ Nauk
SSSR (N. S.)} 78 (1951), 405--408.  Translation: {\em Translations Amer.\ Math.\
Soc.}\ 3 (1962), 287--293.
[Famous theorem asserting that a function continuous
on a compact set in the complex plane whose complement is connected,
and analytic in the interior,
can be uniformly approximated by polynomials.]
\par
{\sc H. N. Mhaskar and D. V. Pai,} {\em Fundamentals of Approximation
Theory,} CRC/Narosa, 2000.
[Extensive treatment of many topics, especially in linear approximation.]
\par
{\sc G. Mittag-Leffler,} Sur la repr\'esentation analytique
des fonctions d'une variable r\'eelle, {\em Rend.\ Circ.\ Mat.\ Palermo}
(1900), 217--224.  [Contains a long footnote by Phragm\'en explaining
how the Weierstrass approximation theorem follows from the
work of Runge.]
\par
{\sc C. Moler and C. Van Loan,} Nineteen dubious ways to compute
the exponential of a matrix, twenty-five years later,
{\em SIAM Rev.}\ 45 (2003), 3--49.  [Expanded reprinting of
1978 paper summarizing methods for computing $\exp(A)$, the best
method being related to Pad\'e approximation.]
\par
{\sc R. de Montessus de Ballore,} Sur les fractions
continues alg\'ebriques, {\em Bull.\ Soc.\ Math.\ France}
30 (1902), 28--36.
[Shows that type $(m,n)$ Pad\'e approximants to meromorphic
functions converge pointwise as $m\to\infty$ in a disk about
$z=0$ with exactly $n$ poles.]
\par
{\sc M. Mori and M. Sugihara,} The double-exponential transformation in
numerical analysis, {\em J. Comput.\ Appl.\ Math.}\ 127 (2001), 287--296.
[Survey of a the quadrature algorithm introduced by Takahasi \& Mori 1974]
\par

\end{par} \vspace{1em}
\begin{par}
 \vspace{-1em} \small \parskip=2pt
\def\parr{{\tiny\sl ~CHECK}\par}
{\sc J.-M. Muller,} {\em Elementary Functions: Algorithms and Implementation},
2nd ed., Birkh\"auser, 2006.  [A text on implementation of
elementary functions on computers, including a chapter on the Remez algorithm.]
\par
{\sc Y. Nakatsukasa, Z. Bai and F. Gygi,} Optimizing Halley's iteration
for computing the matrix polar decomposition, {\em SIAM J. Matrix Anal.\ Appl.}\ 31
(2010), 2700--2720.
[Introduction of an algorithm based on a rational function of
high degree generated by iteration of a simple equiripple approximation.]
\parr
{\sc I. P. Natanson,} {\em Constructive Theory of Functions,} 3 vols., Frederick Ungar,
1964 and 1965.
[This major work by a scholar in Leningrad gives equal emphasis to algebraic
and trigonometric approximation.]
\par
{\sc D. J. Newman,} Rational approximation to $|x|$,
{\em Mich.\ Math.\ J.}\ 11 (1964), 11--14.
[Shows that whereas polynomial approximants to $|x|$ on $[-1,1]$
converge at the rate $O(n^{-1})$, for rational approximants the rate is
$O(\exp(-C\sqrt n\kern 1pt))$.]
\par
{\sc D. J. Newman,} Rational approximation to $e^{-x}$,
{\em J. Approx.\ Th.}\ 10 (1974), 301--303.
[Shows by a lower bound $1280^{-n}$ that type $(n,n)$ rational
approximants to $e^x$ on $(-\infty,0\kern .5pt]$ can converge
no faster than geometrically as $n\to\infty$ in the supremum norm.]
\par
{\sc J. Nuttall,} The convergence of Pad\'e approximants of
meromorphic functions,
{\em J. Math.\ Anal.\ Appl.}\ 31 (1970), 147--153.
[Shows that type $(n,n)$ Pad\'e approximants to meromorphic
functions converge in measure as $n\to\infty$, though not pointwise.]
\par

\end{par} \vspace{1em}
\begin{par}
 \vspace{-1em} \small \parskip=2pt
\def\parr{{\tiny\sl ~CHECK}\par}
{\sc H. O'Hara and F. J. Smith,} Error estimation in the Clenshaw--Curtis quadrature
formula, {\em Comput.\ J.}\ 11 (1968), 213--219.
[Early paper arguing that Clenshaw--Curtis and Gauss quadrature have comparable
accuracy in practice.]
\par
{\sc A. V. Oppenheim, R. W. Schafer and J. R. Buck,}
{\em Discrete-time Signal Processing}, Prentice Hall, 1999.
[A standard textbook on the subject, which is tightly connected with
polynomial and rational approximation.]
\par
{\sc S. A. Orszag (1971a),} Galerkin approximations to flows within slabs, spheres,
and cylinders, {\em Phys.\ Rev.\ Lett.}\ 26 (1971), 1100--1103.
[Orszag's first publication on Chebyshev spectral methods.]
\par
{\sc S. A. Orszag (1971b),} Accurate solution of the Orr--Sommerfeld
stability equation, {\em J. Fluid Mech.}\ 50 (1971), 689--703.
[The most influential of Orszag's early papers on Chebyshev spectral methods.]
\par
{\sc R. Pach\'on,} {\em Algorithms for Polynomial and Rational Approximation
in the Complex Domain,} DPhil thesis, U. of Oxford, 2010.
[Includes chapters on rational best approximants, interpolants, and Chebyshev--Pad\'e
approximants and their application to exploration of functions in the
complex plane.]
\par
{\sc R. Pach\'on, P. Gonnet and J. Van Deun,} Fast and stable rational
interpolation in roots of unity and Chebyshev points, {\em SIAM J. Numer.\ Anal.},
to appear.
[Linear algebra formulation of the rational interpolation problem in a
manner closely suited to computation.]
\par
{\sc R. Pach\'on, R. B. Platte and L. N. Trefethen,}
Piecewise-smooth chebfuns, {\em IMA J. Numer.\ Anal.}\ 30 (2010),
898--916.
[Generalization of Chebfun from single to multiple polynomial pieces,
including edge detection algorithm to determine breakpoints.]
\par
{\sc R. Pach\'on and L. N. Trefethen,} Barycentric-Remez algorithms
for best polynomial approximation in the chebfun system,
{\em BIT Numer.\ Math.}\ 49 (2009), 721--741.  [Chebfun implementation
of Remez algorithm for computing polynomial best approximations.]
\par
{\sc H. Pad\'e,} Sur la repr\'esentation approch\'ee d'une fonction par
des fractions rationelles, {\em Annales Sci.\ de l'\'Ecole Norm.\ Sup.}\ 9
(1892) (suppl\'ement), 3--93.
[The first of many publications by Pad\'e on the subject that
became known as Pad\'e approximation, with discussion of defect
and block structure including a number of explicit examples.]
\par
{\sc T. W. Parks and J. H. McClellan,} Chebyshev approximation for nonrecursive
digital filters with linear phase,
{\em IEEE Trans.\ Circuit Theory}  CT-19 (1972), 189--194.
[Proposes what became known as the
Parks--McClellan algorithm for digital filter design,
based on a barycentric formulation of the Remez
algorithm for best approximation by trigonometric polynomials.]
\par
{\sc B. N. Parlett and C. Reinsch,} Handbook series linear algebra:
balancing a matrix for calculation of eigenvalues and eigenvectors,
{\em Numer.\ Math.}\ 13 (1969), 293--304.  [Introduction of the
technique of balancing a matrix by a diagonal similarity transformation that
is crucial to the success of the QR algorithm.]
\par
{\sc K. Pearson,} {\em On the Construction of Tables and on Interpolation I.
Uni-variate Tables,} Cambridge U. Press, 1920.
[Contains as an appendix a fascinating annotated bibliography of 50 early
contributions to interpolation.  Pearson's annotations are not always as
polite as my own, with comments like ``Not very adequate'' and
``A useful, but somewhat disappointing book.'']
\par
{\sc O. Perron,} {\em Die Lehre von den Kettenbr\"uchen,} 2nd ed., Teubner, 1929.
[This classic monograph on continued fractions, first published in 1913,
was perhaps the first to identify
the problem of spurious poles or Froissart doublets in Pad\'e approximation.
At the end of \S 78 a function is constructed whose type $(m,1)$ Pad\'e approximants
have poles appearing infinitely often on a dense set of points in the complex
plane.]
\par
{\sc P. P. Petrushev and V. A. Popov,} {\em Rational Approximation of
Real Functions,} Cambridge U. Press, 1987.  [Detailed
presentation of a great range of results known up to 1987.]
\par
{\sc R. Piessens,} Algorithm 473: Computation of Legendre series coefficients [C6],
{\em Comm.\ ACM} 17 (1974), 25--25.
[$O(n^2)$ algorithm for converting from Chebyshev to Legendre expansions.]
\par

\end{par} \vspace{1em}
\begin{par}
 \vspace{-1em} \small \parskip=2pt
\def\parr{{\tiny\sl ~CHECK}\par}
{\sc A. Pinkus,} Weierstrass and approximation theory,
{\em J. Approx.\ Th.}\ 107 (2000), 1--66.  [Detailed
discussion of Weierstrass's nowhere-differentiable function
and of the Weierstrass approximation theorem and its
many proofs and generalizations.]
\par
{\sc R. B. Platte, L. N. Trefethen and A. B. J. Kuijlaars,}
Impossibility of fast stable approximation of analytic functions
from equispaced samples, {\em SIAM Rev.}\ 53 (2011), 308--318.
[Shows that any exponentially convergent scheme for approximating
analytic functions from equispaced samples in an interval must be
exponentially ill-conditioned as $n\to\infty$;
thus no approximation scheme can eliminate the Gibbs and Runge phenomena
completely.]
\par
{\sc G. P\'olya,} \"Uber die Konvergenz von Quadraturverfahren,
{\em Math.\ Z.}\ 37 (1933), 264--286.
[Proves that a family of interpolating
quadrature rules converges for all continuous integrands
if and only if the sums of the absolute values of the weights
are uniformly bounded; proves further that
Newton--Cotes quadrature approximations do not
always converge as $n\to\infty$, even if the integrand is analytic.]
\par
{\sc Ch.\ Pommerenke,} Pad\'e approximants and convergence in
capacity, {\em J. Math.\ Anal.\ Appl.}, 41 (1973), 775--780.
[Sharpens Nuttall's result on convergence of Pad\'e approximants
in measure to convergence in capacity.]
\par
{\sc J. V. Poncelet,} Sur la valeur approch\'ee lin\'eaire et rationelle des radicaux de la
forme $\sqrt{a^2+b^2}, \sqrt{a^2-b^2}$ etc., {\em J. Reine Angew.\ Math.}\ 13 (1835),
277--291.
[Perhaps the very first discussion of minimax approximation.]
\par
{\sc D. Potts, G. Steidl and M. Tasche,} Fast algorithms for
discrete polynomial transforms, {\em Math.\ Comp.}\ 67 (1998),
1577--1590. [Algorithms for converting between Chebyshev and
Legendre expansions.]
\par
{\sc M. J. D. Powell,} {\em Approximation Theory and Methods,}
Cambridge U. Press, 1981.  [Approximation theory
text with a computational emphasis, particularly strong on the
Remez algorithm and on splines.]
\par
{\sc H. A. Priestley,} {\em Introduction to Complex Analysis,} 2nd ed.,
Oxford U. Press, 2003.
[Well known introductory complex analysis textbook first published in 1985.]
\par
{\sc I. E. Pritsker and R. S. Varga,} The Szeg\H o curve, zero distribution
and weighted approximation, {\em Trans.\ Amer.\ Math.\ Soc.}\ 349
(1997), 4085--4105.  [Analysis of the Szeg\H o curve using
methods of potential theory.]
\par
{\sc P. Rabinowitz,} Rough and ready error estimates in Gaussian integration
of analytic functions, {\em Comm.\ ACM} 12 (1969), 268--270.
[Derives tight bounds on accuracy of Gaussian quadrature by simple arguments.]
\par
{\sc H.-J. Rack and M. Reimer,} The numerical stability of evaluation
schemes for polynomials based on the Lagrange interpolation form,
{\em BIT} 22 (1982), 101--107.  [Proof of stability
for barycentric polynomial interpolation in well-distributed point sets,
later developed further by Higham 2004.]
\par
{\sc T. Ransford,} {\em Potential Theory in the Complex Plane,} Cambridge U. Press,
1995.  [Perhaps the only book devoted to this subject.]
\par
{\sc T. Ransford,} Computation of logarithmic capacity,
{\em Comput.\ Meth.\ Funct.\ Th.}\ 10 (2010), 555--578.
[An algorithm for computing capacity of a set in the complex plane, with examples.]
\par
{\sc E. Remes,} Sur un proc\'ed\'e convergent d'approximations successives
pour d\'eterminer les polyn\^omes d'approximation, {\em Compt.\ Rend.\ Acad.\ Sci.}\ 198
(1934), 2063--2065.  [One of the original papers presenting the Remez algorithm.]
\parr
{\sc E. Remes,} Sur le calcul effectif des polyn\^omes d'approximation de
Tchebichef, {\em Compt.\ Rend.\ Acad.\ Sci.}\ 199 (1934), 337--340.
[The other original paper presenting the Remez algorithm.]
\parr
{\sc E. Y. Remes,} On approximations in the complex domain,
{\em Dokl.\ Akad.\ Nauk SSSR} 77 (1951), 965--968 [Russian].
\parr
{\sc E. Ya.\ Remez,} General computational methods of Tchebycheff approximation,
Atomic Energy Commission Translation 4491, Kiev, 1957, pp.~1--85.
\parr
{\sc J. R. Rice,} {\em The Approximation of Functions,}
Addison-Wesley, 1964 and 1969.
[Two volumes, the first linear and the second nonlinear.]
\par
{\sc L. F. Richardson,} The deferred approach to the limit. I---single lattice.
{\em Phil.\ Trans.\ Roy.\ Soc.\ A} (1927), 299--349.
[Systematic discussion of Richardson extrapolation, emphasizing discretizations
with $O(h^2)$ error behavior.]
\par
{\sc M. Richardson and L. N. Trefethen,} A sinc function analogue
of Chebfun, {\em SIAM J. Sci.\ Comput.}\ 33 (2011), 2519--2535.
[Presents a ``Sincfun'' software analogue of Chebfun for dealing with
functions with endpoint singularities via variable transformation and
sinc function interpolants.]
\par
{\sc F. Riesz,} \"Uber lineare Funktionalgleichungen, {\em Acta Math.}\ 41
(1918), 71--98.  [First statement of the general existence result for
best approximation from finite-dimensional linear spaces.]
\par
{\sc M. Riesz,} \"Uber einen Satz des Herrn Serge Bernstein,
{\em Acta. Math.}\ 40 (1916), 43--47.  [Gives a new proof of
a Bernstein inequality
based on the barycentric formula for Chebyshev points, in the process
deriving the barycentric coefficients
$(-1)^j$ half a century before Salzer 1972.]
\parr
{\sc T. J. Rivlin,} {\em An Introduction to the Approximation of
Functions,} Dover, 1981.
[Appealing short textbook originally published in 1969.]
\par
{\sc T. J. Rivlin,} {\em Chebyshev Polynomials: From Approximation Theory
to Algebra and Number Theory,} 2nd ed., Wiley, 1990.
[Classic book on Chebyshev polynomials and applications, with first
edition in 1974.]
\par
{\sc J. D. Roberts,} Linear model reduction and solution of
the algebraic Riccati equation by use of the sign function,
{\em Internat.\ J. Control} 32 (1980), 677--687.
[This article connecting rational functions with the sign function
was written in 1971 as Technical Report CUED/B-Control/TR13 of
the Cambridge University Engineering Dept.]
\par
{\sc W. Rudin,} {\em Principles of Mathematical Analysis,} 3rd ed.,
McGraw-Hill, 1976.  [Influential textbook first published in 1953.]
\par
{\sc P. O. Runck,} \"Uber Konvergenzfragen bei
Polynominterpolation mit \"aquidistanten Knoten.\ II,
{\em J. Reine Angew.\ Math.}\ 210 (1962), 175--204.
[Analyzes the Gibbs overshoot for two varieties of polynomial
interpolation of a step function.]
\par
{\sc C. Runge (1885a),} Zur Theorie der eindeutigen analytischen Functionen,
{\em Acta Math.}\ 6 (1885), 229--244.
[Publication of Runge's theorem:
a function analytic on a compact set in the complex plane whose complement
is connected can be uniformly approximated by polynomials.  This was
later generalized by Mergelyan.]
\par
{\sc C. Runge (1885b),} \"Uber die Darstellung willk\"urlicher Functionen,
{\em Acta Math.}\ 7 (1885), 387--392.  [Shows that a continuous function on
a finite interval can be uniformly approximated by rational functions.
It was later noted by Phragm\'en and Mittag-Leffler that
this and the previous paper by Runge together imply the Weierstrass
approximation theorem.]
\par

\end{par} \vspace{1em}
\begin{par}
 \vspace{-1em} \small \parskip=2pt
\def\parr{{\tiny\sl ~CHECK}\par}
{\sc C. Runge,} \"Uber empirische Funktionen und die Interpolation
zwischen \"aquidistanten Ordinaten,
{\em Z. Math.\ Phys.}\ 46 (1901), 224--243.
[M\'eray had pointed out that polynomial interpolants
might fail to converge, but it was this paper that focussed on equispaced
sample points, showed that divergence can take place even in the
interval of interpolation, and identified the ``Runge region'' where
analyticity is required for convergence.]
\par
{\sc A. Ruttan,} The length of the alternation set as a factor in
determining when a best real rational approximation is also a best complex
rational approximation, {\em J. Approx.\ Th.}\ 31 (1981), 230--243.
[Shows that complex best approximations are always better
than real ones in the strict lower-right triangle of a square block of
the Walsh table.]
\par
{\sc A. Ruttan and R. S. Varga,} A unified theory for real vs.\ complex
rational Chebyshev approximation on an interval, {\em Trans.\ Amer.\ Math.\ Soc.}\ 312 (1989),
681--697.  [Shows that type $(m,m+2)$ complex rational approximants to real functions
can be up to $3$ times as accurate as real ones.]
\par
{\sc E. B. Saff,} An extension of Montessus de Ballore's theorem
on the convergence of interpolating rational functions, {\em J. Approx.\ Th.}\ 6
(1972), 63--67.  [Generalizes the de Montessus de Ballore theorem from
Pad\'e to multipoint Pad\'e approximation.]
\par
{\sc E. B. Saff and A. D. Snider,} {\em Fundamentals of Complex
Analysis with Applications to Engineering, Science, and Mathematics,}
3rd ed., Prentice Hall, 2003.
[Widely used introductory complex analysis textbook.]
\par
{\sc E. B. Saff and V. Totik,} {\em Logarithmic Potentials with
External Fields,} Springer, 1997.
[Presentation of connections between potential theory and rational approximation.]
\par
{\sc E. B. Saff and R. S. Varga (1978a),} Nonuniqueness of best complex
rational approximations to real functions on real intervals,
{\em J. Approx.\ Th.}\ 23 (1978), 78--85.  [Rediscovery of results of
Lungu 1971.]
\par
{\sc E. B. Saff and R. S. Varga} (1978b), On the zeros and poles
of Pad\'e approximants to $e^z$. III,
{\em Numer.\ Math.}\ 30 (1978), 241--266.
[Analysis of the curves in the complex plane along which poles and
zeros of these approximants cluster.]
\par
{\sc T. W. Sag and G. Szekeres,} Numerical evaluation of high-dimensional
integrals, {\em Math.\ Comp.}\ 18 (1964), 245--253.  [Introduction
of changes of variables that can speed up Gauss and other quadrature
formulas, even in one dimension.]
\par
{\sc Salazar Celis,}
\parr
{\sc H. E. Salzer,} A simple method for summing certain slowly
convergent series, {\em J. Math.\ Phys.}\ 33 (1955), 356--359.
[``Salzer's method'' for acceleration of convergence, based on
interpreting a sequence of values as samples of a function $f(x)$
at $x_n=n^{-1}$.]
\par
{\sc H. E. Salzer,} Lagrangian interpolation at the Chebyshev points
$x_{n,\nu} = \cos(\nu \pi/n)$, $\nu = 0(1)n$; some unnoted advantages,
{\em Computer J.}\ 15 (1972), 156--159.  [Barycentric formula
for polynomial interpolation in Chebyshev points.]
\par
{\sc H. E. Salzer,} Rational interpolation using incomplete barycentric forms,
{\em Z. Angew.\ Math.\ Mech.}\ 61 (1981), 161--164.
[One of the first publications to propose the use of
rational interpolants defined by barycentric formulas.]
\par
{\sc T. Schmelzer and L. N. Trefethen,} Evaluating matrix functions
for exponential integrators via
Carath\'eodory--Fej\'er approximation and
contour integrals, {\em Elect.\ Trans.\ Numer.\ Anal.}\ 29 (2007), 1--18.
[Fast methods based on rational approximations
for evaluating the $\varphi$ functions used by
exponential integrators for solving stiff ODEs and PDEs.]
\par
{\sc J. R. Schmidt,} On the numerical solution of linear
simultaneous equations by an iterative method,
{\em Philos.\ Mag.} 32 (1941),  369--383.
[Proposal of what became known as the epsilon or eta algorithm some years
before Shanks 1955, Wynn 1956, and Bauer 1959.]
\par
{\sc C. Schneider and W. Werner,} Some new aspects of rational interpolation,
{\em Math.\ Comp.}\ 47 (1986), 285--299.
[Extension of barycentric formulas to rational interpolation.]
\par
{\sc A. Sch\"onhage,} Fehlerfortpflanzung bei Interpolation,
{\em Numer.\ Math.}\  3 (1961), 62--71.
[Independent rediscovery of results close to those of Turetskii 1940
concerning Lebesgue constants for equispaced points.]
\par
{\sc A. Sch\"onhage,} Zur rationalen Approximierbarkeit von
$e^{-x}$ \"uber $[0,\infty)$, {\em J. Approx.\ Th.}\ 7 (1973), 395--398.
[Proves that in maximum-norm approximation of $e^x$ on $(-\infty,0\kern .5pt]$ by
inverse-polynomials $1/p_n(x)$, the optimal rate is $O(3^{-n})$.]
\par
{\sc I. Schur,} \"Uber Potenzreihen, die im Innern des Einheitskreises
beschr\"ankt sind, {\em J. Reine Angew.\ Math.}\ 148 (1918), 122--145.
[Solution of the problem of Carath\'eodory and Fej\'er via the
eigenvalue analysis of a Hankel matrix of Taylor coefficients.]
\par
{\sc D. Shanks,} Non-linear transformations of divergent and slowly
convergent sequences, {\em J. Math.\ Phys.}\ 34 (1955), 1--42.
[Introduction of Shanks' method for convergence acceleration by
Pad\'e approximation, closely related to the epsilon algorithm of
Wynn 1956.]
\par
{\sc J. Shen, T. Tang and L.-L. Wang,} {\em Spectral Methods: Algorithms, Analysis
and Applications,} Springer, 2011.
[Systematic presentation of spectral methods including convergence theory.]
\par
{\sc B. Shiffman and S. Zelditch,} Equilibrium distribution of zeros of
random polynomials, {\em Int.\ Math.\ Res.\ Not.} 2003, no.~1.
[Shows that polynomials given by expansions
in orthogonal polynomials with random
coefficients have roots clustering
near the support of the orthogonality measure.]
\parr
{\sc A. Sidi,} {\em Practical Extrapolation Methods,} Cambridge U. Press, 2003.
[Extensive treatment of methods for acceleration of convergence.]
\par
{\sc G. A. Sitton, C. S. Burrus, J. W. Fox and S. Treitel,}
Factoring very-high-degree polynomials, {\em IEEE Signal
Proc.\ Mag.}, Nov.\ 2003, 27--42.
[Discussion of rootfinding for polynomials of degree up to
one million by the Lindsey--Fox algorithm.]
\par
{\sc V. I. Smirnov and N. A. Lebedev,} {\em Functions of a Complex
Variable: Constructive Theory}, MIT Press, 1968.
[Major survey of problems of polynomial and rational approximation
in the complex plane.]
\par
{\sc F. Smithies,} {\em Cauchy and the Creation of Complex Function
Theory}, Cambridge U. Press, 1997.  [Detailed account of Cauchy's
almost single-handed creation of this field during 1814--1831.]
\par
{\sc M. A. Snyder,} {\em Chebyshev Methods in Numerical
Approximation,} Prentice Hall, 1966.
[An appealing short book emphasizing rational as well as
polynomial approximations.]
\par
{\sc W. Specht,} Die Lage der Nullstellen eines Polynoms. III,
{\em Math.\ Nachr.}\ 16 (1957), 369--389.
[Development of comrade matrices, whose eigenvalues are roots
of polynomials expressed in bases of orthogonal polynomials.]
\par
{\sc W. Specht,} Die Lage der Nullstellen eines Polynoms. IV,
{\em Math.\ Nachr.}\ 21 (1960), 201--222.
[The final page considers colleague matrices, the special
case of comrade matrices for
Chebyshev polynomials.  These ideas were developed independently
by Good 1961.]
\par
{\sc H. Stahl,} The convergence of Pad\'e approximants
to functions with branch points, {\em J. Approx.\ Th.}\ 91
(1997), 139--204.
[Generalizes the Nuttall--Pommerenke theorem on convergence of
type $(n,n)$ Pad\'e approximants to the case of functions $f$ with
branch points.]
\par
{\sc H. Stahl,} Spurious poles in Pad\'e approximation,
{\em J. Comp.\ Appl.\ Math.}\ 99 (1998), 511--527.
[Defines and analyzes what it means for a pole of a Pad\'e approximant
to be spurious.]
\par

\end{par} \vspace{1em}
\begin{par}
 \vspace{-1em} \small \parskip=2pt
\def\parr{{\tiny\sl ~CHECK}\par}
{\sc H. Stahl,} Best uniform rational approximation of $|x|$ on $[-1,1]$,
{\em Russian Acad.\ Sci.\ Sb.\ Math.}\  76 (1993), 461--487.
[Proof of the conjecture of Varga, Ruttan and Carpenter that
best rational approximations to $|x|$ on $[-1,1]$
converge at the rate $\sim 8\kern .1pt \exp(-\pi \sqrt n\kern .9pt)$.]
\par
{\sc H. R. Stahl,} Best uniform rational approximation of $x^\alpha$ on $[0,1]$,
{\em Acta Math.}\ 190 (2003), 241--306.
[Generalization of the results of the paper above to approximation of
$x^\alpha$ on $[0,1]$, completing earlier
investigations of Ganelius and Vyacheslavov.]
\par
{\sc H. Stahl and T. Schmelzer,} An extension of the `1/9'-problem,
{\em J. Comput.\ Appl.\ Math.}\ 233 (2009), 821--834.
[Announcement of numerous extensions of the ``9.28903'' result
of Gonchar \& Rakhmanov 1989 for
type $(n,n)$ best approximation of $e^x$ on $(-\infty,0\kern .5pt]$,
showing that the same rate of approximation applies on compact sets in
the complex plane and on Hankel contours, and that ``9.28903'' is also
achieved on $(-\infty,0\kern.5pt]$
in type $(n,n+k)$ approximation of $e^x$ or of
related functions such as $\varphi$ functions for exponential integrators.]
\par
{\sc K.-G. Steffens,} {\em The History of Approximation Theory: From Euler
to Bernstein}, Birkh\"auser, 2006.
[Discussion of many people and results, especially of the St.\ Petersburg
school, by a student of Natanson.]
\par
{\sc E. M. Stein and R. Shakarchi,} {\em Real Analysis: Measure Theory,
Integration, and Hilbert Spaces,} Princeton U. Press, 2005.
[A leading textbook.]
\par
{\sc F. Stenger,} Explicit nearly optimal linear rational approximation
with preassigned poles, {\em Math.\ Comput.}\ 47 (1986), 225--252.
[Construction of rational approximants by a method related to sinc
expansions.]
\par
{\sc F. Stenger,} {\em Numerical Methods Based on Sinc and
Analytic Functions,} Springer, 1993.  [Comprehensive treatise by the
leader in sinc function algorithms.]
\par
{\sc F. Stenger,} {\em Sinc Numerical Methods,} CRC Press, 2010.
[A handbook of sinc methods and their implementation in
the author's software package Sinc-Pack.]
\par
{\sc T. J. Stieltjes (1884a),} Note sur quelques formules pour l'\'evaluation de certaines
int\'egrales, {\em Bull.\ Astr.\ Paris} 1 (1884), 568--569.
\parr
{\sc T. J. Stieltjes (1884b),} Quelques recherches sur la th\'eorie des
quadratures dites m\'ecaniques, {\em Ann.\ Sci.\ \'Ecole Norm.\ Sup.}\
1 (1884), 409--426.
[Proves that Gauss quadrature converges for any
Riemann integrable integrand.]
\parr
{\sc T. J. Stieltjes,}  Sur les polyn\^omes de
Jacobi, {\em Compt.\ Rend.\ Acad.\ Sci.}\ 199 (1885), 620--622.
[Shows that the roots of $(x^2-1)P_{n-1}^{(1,1)}(x)$ are
Fekete points (minimal-energy points) in $[-1,1]$.]
\par
{\sc J. Szabados,} Rational approximation to analytic functions
on an inner part of the domain of analyticity, in A. Talbot, ed.,
{\em Approximation Theory,} Academic Press, 1970, pp.\ 165--177.
[Shows that for some
functions analytic in a Bernstein $\rho$-ellipse, type $(n,n)$ rational best
approximations are essentially no better than degree $n$ polynomial
best approximations.]
\par
{\sc G. Szeg\H o,} \"Uber eine Eigenschaft der Exponentialreihe,
{\em Sitzungsber.\ Berl.\ Math.\ Ges.}\ 23 (1924), 50--64.
[Shows that as $n\to\infty$, the zeros of the
normalized partial sums $s_n(nz)$ of the Taylor series of $e^z$ approach
the Szeg\H o curve in the complex $z$-plane
defined by $|z\kern .7pt e^{1-z}|=1$ and $|z|\le 1$.]
\par
{\sc G. Szeg\H o,} {\em Orthogonal Polynomials,} Amer.\ Math.\ Soc., 1985.
[A classic monograph by
the master, including chapters on polynomial interpolation and quadrature, first
published in 1939.]
\par
{\sc E. Tadmor,} The exponential accuracy of Fourier and Chebyshev
differencing methods, {\em SIAM J. Numer.\ Anal.}\ 23 (1986), 1--10.
[Presents theorems on
exponential accuracy of Chebyshev interpolants of analytic functions
and their derviatives.]
\par
{\sc T. Takagi,} On an algebraic problem related to an analytic
theorem of Carath\'eodory and Fej\'er and on an
allied theorem of Landau, {\em Japan J. Math.}\ 1 (1924), 83--91 and
ibid., 2 (1925), 13--17.  [Beginnings of the generalization
of Carath\'eodory \& Fej\'er 1911 and Schur 1918 to rational approximation.]
\par
{\sc H. Takahasi and M. Mori,} Estimation of errors in the numerical
quadrature of analytic functions, {\em Applicable Anal.}\ 1 (1971), 201--229.
[Relates the accuracy of a quadrature formula to the accuracy of an associated
rational function as an approximation to $\log((z+1)/(z-1))$ on a
contour enclosing $[-1,1]$.]
\par
{\sc H. Takahasi and M. Mori,} Double exponential formulas for numerical
integration, {\em Publ.\ RIMS, Kyoto U.}\ 9 (1974), 721--741.
[Introduction of the double exponential or tanh-sinh quadrature rule,
in which Gauss quadrature is transformed by a change of variables to
another formula that can handle endpoint singularities.]
\par
{\sc A. Talbot,} The uniform approximation of polynomials by polynomials of
lower degree, {\em J. Approx.\ Th.}\ 17 (1976), 254--279.
[A precursor to the Carath\'eodory--Fej\'er method.]
\par
{\sc F. D. Tappert,} The parabolic approximation method, in J. B. Keller and J. S.
Papadakis, eds., {\em Wave Propagation and Underwater Acoustics,}
Springer, 1977, pp.~224--287.
[Describes techniques for one-way acoustic wave simulation in
the ocean, based on polynomial and rational approximations of
a pseudodifferential operator.]
\par
{\sc R. Taylor and V. Totik,} Lebesgue constants for Leja points,
{\em IMA J. Numer.\ Anal.}\ 30 (2010), 462--486.
[Proves that for general sets in the complex plane, the Lebesgue
constants associated with Leja points grow subexponentially.]
\par
{\sc W. J. Taylor,} Method of Lagrangian curvilinear interpolation,
{\em J. Res.\ Nat.\ Bur.\ Stand.}\ 35 (1945), 151--155.  [The first
use of the barycentric interpolation formula, for equidistant points only and
without the term ``barycentric'', which was introduced by Dupuy 1948.]
\par
{\sc T. W. Tee and L. N. Trefethen,} A rational spectral collocation
method with adaptively transformed Chebyshev grid points,
{\em SIAM J. Sci.\ Comp.}\ 28 (2006), 1798--1811.
[Numerical solution of differential equations with highly nonuniform
solutions using Chebyshev--Pad\'e approximation, conformal maps, and
spectral methods based on rational barycentric interpolants, as advocated
by Berrut and coauthors.]
\par
{\sc H. Tietze,} Eine Bemerkung zur Interpolation, {\em Z. Angew.\ Math.\ Phys.}\ 64
(1917), 74--90.  [Investigates the Lebesgue function for equidistant points,
showing the local maxima decrease monotonically from the outside of the interval
toward the middle.]
\par
{\sc A. F. Timan,} A strengthening of Jackson's theorem on the
best approximation of continuous functions by polynomials on a finite
interval of the real axis, {\em Doklady Akad.\ Nauk SSSR} 78 (1951), 17--20.
[A theorem on polynomial approximation that recognizes the greater
approximation power near the ends of the interval.]
\par
{\sc A. F. Timan,} {\em Theory of Approximation of Functions of a Real
Variable,} Dover, 1994.  [First published in Russian in 1960.]
\par
{\sc K.-C. Toh and L. N. Trefethen,} Pseudozeros of polynomials and
pseudospectra of companion matrices, {\em Numer.\ Math.}\ 68 (1994), 403--425.
[Analysis of stability of companion matrix eigenvalues as
an algorithm for polynomial rootfinding, given a polynomial
expressed by its coefficients in the monomial basis.]
\par
{\sc L. Tonelli,} I polinomi d'approssimazione di Tschebychev, {\em Annali di Mat.}\ 15
(1908), 47--119.  [Extension of results on real best approximation
to the complex case.]
\par
{\sc L. N. Trefethen,} Chebyshev approximation on the unit disk, in
H. Werner et al., eds., {\em Constructive Aspects of Complex
Analysis}, D. Riedel, 1983.  [An introduction
to several varieties of CF approximation.]
\par
{\sc L. N. Trefethen,} Square blocks and equioscillation
in the Pad\'e, Walsh, and CF tables, in
P. R. Graves-Morris, et al., eds., {\em Rational Approximation
and Interpolation,} Lect.\ Notes in Math., v.~1105, Springer, 1984.
[Shows that square block structure in all three tables
of rational approximations arises from equioscillation-type
characterizations involving the defect.]
\par

\end{par} \vspace{1em}
\begin{par}
 \vspace{-1em} \small \parskip=2pt
\def\parr{{\tiny\sl ~CHECK}\par}
{\sc L. N. Trefethen,} {\em Spectral Methods in MATLAB}, SIAM, 2000.
[Matlab-based textbook on spectral methods for ODEs and PDEs.]
\par
{\sc L. N. Trefethen,} Is Gauss quadrature better than
Clenshaw--Curtis?, {\em SIAM Rev.}\ 50 (2008), 67--87.
[Shows by considering approximation properties in the complex plane
that for most functions, the Clenshaw--Curtis and Gauss formulas
have comparable accuracy.]
\par
{\sc L. N. Trefethen,} Householder triangularization of a quasimatrix,
{\em IMA J. Numer.\ Anal.}\ 30 (2010), 887--897.
[Extends the Householder triangularization algorithm to quasimatrices, i.e.,
``matrices'' whose columns are functions rather than vectors.]
\par
{\sc L. N. Trefethen and D. Bau,} III,
{\em Numerical Linear Algebra,} SIAM, 1997.
[A standard text, with a section ``When vectors become continuous
functions'' at p.~52 that foreshadows Chebfun computation with quasimatrices.]
\par
{\sc L. N. Trefethen and M. H. Gutknecht (1983a),} Real vs.\ complex rational Chebyshev
approximation on an interval, {\em Trans.\ Amer.\ Math.\ Soc.}\ 280
(1983), 555--561.
[Shows that type $(m,n)$ complex rational approximations to a real
function on an interval may be arbitrarily much better than real
ones, for $n\ge m+3$.]
\par
{\sc L. N. Trefethen and M. H. Gutknecht (1983b),}
The Carath\'eodory--Fej\'er method for real rational
approximation, {\em SIAM J. Numer.\ Anal.}\ 20 (1983), 420--436.
[Introduction of real rational CF approximation, and first numerical computation
of the constant $9.28903\dots$ for minimax rational approximation of $e^x$ on
$(-\infty,0\kern .7pt ]$.]
\par
{\sc L. N. Trefethen and M. H. Gutknecht,}
On convergence and degeneracy in rational Pad\'e and Chebyshev
approximation, {\em SIAM J. Math.\ Anal.}\ 16 (1985), 198--210.
[Proves theorems to the effect that the Pad\'e approximation operator
is continuous, and Pad\'e approximants are limits of best approximants on
regions shrinking to a point, provided that the defect is $0$.]
\par
{\sc L. N. Trefethen and M. H. Gutknecht,} Pad\'e, stable Pad\'e, and
Chebyshev--Pad\'e approximation, in J. C. Mason and M. G. Cox,
{\em Algorithms for Approximation,} Clarendon Press, 1987, pp.~227--264.
[Reduces the problem of Chebyshev--Pad\'e approximation to
the problem of stable Pad\'e approximation,
that is, Pad\'e approximation with a constraint on location of poles.]
\par
{\sc L. N. Trefethen and L. Halpern,} Well-posedness of
one-way wave equations and absorbing boundary conditions,
{\em Math.\ Comput.}\ 47 (1986), 421--435.
[Shows that approximations from two diagonals of the Pad\'e table must
be used in these applications; polynomial and other approximations are ill-posed.]
\par
{\sc L. N. Trefethen and J. A. C. Weideman,} Two results concerning polynomial
interpolation in equally spaced points, {\em J. Approx.\ Th.}\ 65 (1991), 247--260.
[Discussion of the size of Lebesgue constants and ``6 points per wavelength''
for polynomial interpolation in equispaced points.]
\par
{\sc L. N. Trefethen, J. A. C. Weideman and T. Schmelzer,} Talbot quadratures
and rational approximations, {\em BIT Numer.\ Math.}\ 46 (2006), 653--670.
[Shows how integrals approximated by the trapezoid rule correspond to
rational approximations in the complex plane, with particular attention to the
approximation of $e^x$ on $(-\infty,0\kern .5pt ]$.]
\par
{\sc A. H. Turetskii,} The bounding of polynomials prescribed at equally
distributed points, {\em Proc.\ Pedag.\ Inst.\ Vitebsk} 3 (1940), 117--127 (Russian).
[Derivation of the $\sim 2^n/e\kern .7pt n \log n$ asymptotic size of Lebesgue constants for
equispaced polynomial interpolation.  This paper went largely unnoticed
for fifty years and the main result was
rediscovered by Sch\"onhage 1961.]
\par
{\sc Ch.-J. de la Vall\'ee Poussin,} Note sur l'approximation par un
polyn\^ome d'une fonction dont la deriv\'ee est
\`a variation born\'ee, {\em Bull.\ Acad.\ Belg.}\ 1908, 403--410.
\parr
{\sc Ch.\ de la Vall\'ee Poussin,} Sur les polyn\^omes d'approximation et
la repr\'esentation approch\'ee d'un angle, {\em Acad.\ Roy.\ de
Belg., Bulletins de la Classe des Sci.}\ 12 (1910).
\parr
{\sc Ch.\ J. de la Vall\'ee Poussin,} {\em Le\c cons sur l'approximation des fonctions
d'une variable r\'eelle,} Gauthier-Villars, Paris, 1919.
\parr
{\sc J. Van Deun and L. N. Trefethen,} A robust implementation of
the Carath\'eodory--Fej\'er method, {\em BIT Numer.\ Math.}\
51 (2011), 1039--1050.
[Twenty-five years after the original theoretical papers, a paper
describing the practical details behind the Chebfun {\tt cf}
command.]
\par
{\sc R. S. Varga and A. J. Carpenter,} On the Bernstein conjecture
in approximation theory, {\em Constr.\ Approx.}\ 1 (1985), 333--348.
[Shows that degree $n$ best polynomial approximants to $|x|$ have asymptotic
accuracy $0.280\dots n^{-1}$ rather than $0.282\dots n^{-1}$.]
\par
{\sc R. S. Varga, A. Ruttan and A. J. Carpenter,} Numerical results
on best uniform rational approximation of $|x|$ on $[-1,1]$,
{\em Math.\ USSR Sbornik} 74 (1993), 271--290.  [High-precision
numerical calculations lead to the conjecture
that best rational approximations to $|x|$ on $[-1,1]$
converge asymptotically at the rate
$\sim 8\exp(-\pi \sqrt n\kern 1pt )$, proved by
Stahl 1993.]
\par
{\sc N. S. Vyacheslavov,} On the uniform approximation of $|x|$ by
rational functions, {\em Sov.\ Math.\ Dokl.}\ 16 (1975), 100--104.
[Sharpens the result of Newman 1964 by showing that rational approximations
to $|x|$ on $[-1,1]$
converge at the rate $O(\exp(-\pi \sqrt n\kern .7pt ))$.]
\par
{\sc J. Waldvogel,} Fast construction of the Fej\'er and Clenshaw--Curtis quadrature rules,
{\em BIT Numer.\ Math.}\ 46 (2006), 195--202.
[Presentation of $O(n\log n)$ algorithms for finding nodes and weights.]
\par
{\sc H. Wallin,} On the convergence theory of Pad\'e approximants, in
{\em Linear Operators and Approximation}, Internat.\ Ser.\ Numer.\ Math.\ 20
(1972), pp.~461--469.  [Shows that there exists an entire function $f$
whose $(n,n)$ Pad\'e approximants are unbounded for all $z\ne 0$.]

\end{par} \vspace{1em}
\begin{par}
 \vspace{-1em} \small \parskip=2pt
\def\parr{{\tiny\sl ~CHECK}\par}
{\sc J. L. Walsh,} The existence of rational functions of best approximation,
{\em Trans.\ Amer.\ Math.\ Soc.}\ 33 (1931), 668--689.
[Shows that there exists a best rational approximation of type $(m,n)$ to
a given continuous function $f$, not just on an interval such as
$[-1,1]$ but also on more general sets in the complex plane.]
\par
{\sc J. L. Walsh,} On approximation to an analytic function by rational
functions of best approximation, {\em Math.\ Z.}\ 38 (1934), 163--176.
[Perhaps the first discussion of what is now called the Walsh table, the
table of best rational approximations to a given function $f$ for various types
$(m,n)$.]
\par
{\sc J. L. Walsh,} The analogue for maximally convergent polynomials of
Jentzsch's theorem, {\em Duke Math.\ J.}\ 26 (1959), 605--616.
[Shows that every point on the boundary of a region of convergence of a sequence
of polynomial approximations is the limit of zeros of its partial sums.]
\par
{\sc J. L. Walsh,} {\em Interpolation and Approximation by Rational Functions in the Complex Domain},
5th ed., American Mathematical Society, 1969.
[An encyclopedic but hard-to-read treatise on all kinds of
material related to polynomial and rational approximation in
the complex plane, originally published in 1935.]
\par
{\sc H. Wang and S. Xiang,} On the convergence rates of Legendre approximation,
{\em Math.\ Comp.}\ 81 (2012), 861--877.
[Theorem 3.1 connects barycentric interpolation weights $\{\lambda_k\}$
and Gauss--Legendre quadrature weights $\{w_k\}$.]
\par
{\sc R. C. Ward,} Numerical computation of the matrix
exponential with accuracy estimate, {\em SIAM J. Numer.\ Anal.}\ 14
(1977), 600--610.  [Presentation of a scaling-and-squaring
algorithm for computing the exponential of a matrix by Pad\'e approximation,
a form of which is used by Matlab's {\tt expm} command.]
\par
{\sc E. Waring,} Problems concerning interpolations,
{\em Phil.\ Trans.\ R. Soc.}\ 69 (1779), 59--67.
[Presents the Lagrange interpolation formula 16 years before Lagrange.]
\par
{\sc G. A. Watson,} {\em Approximation Theory and Numerical Methods,}
Wiley, 1980. [Textbook with special attention to $L_1$ and $L_p$ norms.]
\par
{\sc M. Webb,} Computing complex singularities of differential equations
with Chebfun, {\em SIAM Undergrad.\ Research Online,} submitted, 2012.
[Exploration of rational approximation for locating complex singularities
of numerical solutions to ODE problems
including Lorenz and Lotka--Volterra equations.]
\par
{\sc M. Webb, L. N. Trefethen and P. Gonnet,} Stability of barycentric interpolation
formulas, {\em SIAM J. Sci.\ Comp.}, submitted, 2011.
[Confirming the theory of Higham 2004, shows that the ``type 2'' barycentric
interpolation formula can be dangerously unstable if used for
extrapolation outside the data interval.]
\par
{\sc J. A. C. Weideman,} Computing the dynamics of
complex singularities of nonlinear PDEs, {\em SIAM J.
Appl.\ Dyn.\ Syst.}\ 2 (2003), 171--186.
[Applies Pad\'e approximation to computed solutions of nonlinear
time-dependent PDEs to estimate locations of moving poles and other singularities.]
\par
{\sc J. A. C. Weideman and S. C. Reddy,} A MATLAB differentiation matrix
suite, {\em ACM Trans.\ Math.\ Softw.}\ 26 (2000), 465--519.
[A widely-used collection of Matlab programs for generating
Chebyshev, Legendre, Laguerre, Hermite, Fourier, and
sinc spectral differentiation matrices of arbitrary order.]
\par
{\sc J. A. C. Weideman and L. N. Trefethen,}
The kink phenomenon in Fej\'er and Clenshaw--Curtis quadrature,
{\em Numer.\ Math.}\ 107 (2007), 707--727.  [Analysis of the effect that
as $n$ increases, Clenshaw--Curtis quadrature initially converges at the same
rate as Gauss rather than half as fast as commonly supposed.]
\par
{\sc J. A. C. Weideman and L. N. Trefethen,}
Parabolic and hyperbolic contours for computing the Bromwich
integral, {\em Math.\ Comput.}\ 76 (2007), 1341--1356.
[Derivation of geometrically-convergent ``Talbot contour'' type
rational approximations for problems related to $e^x$ on $(-\infty,0\kern .7pt]$.]
\par
{\sc K. Weierstrass,} \"Uber continuierliche Functionen eines reellen
Arguments, die f\"ur keinen Werth des letzteren einen bestimmten
Differentialquotienten besitzen, {\em K\"onigliche
Akademie der Wissenschaften,} 1872.  [Weierstrass's publication of
an example (which he had lectured on a decade earlier) of a continuous,
nowhere-differentiable function.]
\par
{\sc K. Weierstrass,} \"Uber die analytische Darstellbarkeit
sogenannter willk\"urlicher Functionen einer reellen
Ver\"anderlichen, {\em Sitzungsberichte der Akademie zu Berlin},
633--639 and 789--805, 1885.
[Presentation of the Weierstrass approximation theorem.]
\par
{\sc B. D. Welfert,} Generation of pseudospectral differentiation
matrices.\ I, {\em SIAM J. Numer.\ Anal.}\ 34 (1997), 1640--1657.
[Derivation of stable recursive formulas for computation
of derivatives of interpolants.]
\par
{\sc E. J. Weniger,} Nonlinear sequence transformations for the
acceleration of convergence and the summation of divergent series,
{\em Computer Phys.\ Rep.}\ 10 (1989), 189--371 (also available
as arXiv:math/0306302v1, 2003).
[Extensive survey.]
\par
{\sc H. Werner,} On the rational Tschebyscheff operator, {\em Math.\ Z.}\ 86
(1964), 317--326.  [Shows that the operator mapping a real function
$f\in C[-1,1]$ to its best real rational approximation of type $(m,n)$
is continuous if and only if $f$ is itself rational of type $(m,n)$ or
its best approximation has defect 0 (``nondegenerate'').]
\par
{\sc H. Wilbraham,} On a certain periodic function, {\em Cambridge and Dublin Math.\ J.}
3 (1848), 198--201.  [Analyzes the Gibbs phenomenon fifty years before Gibbs.]
\par
{\sc J. H. Wilkinson,} The perfidious polynomial, in G. H. Golub, ed.,
{\em Studies in Numerical Analysis}, Math.\ Assoc.\ Amer., 1984.
[Wilkinson's major work on polynomials was in the 1960s, but this
entertaining review, which won the Chauvenet Prize of the Mathematical
Association of America in 1987,
remains noteworthy not least because of its memorable title.]
\par
{\sc J. Wimp,} {\em Sequence Transformations and their Applications,}
Academic Press, 1981.
[Monograph on many methods for acceleration of convergence.]
\par
{\sc C. Winston,} On mechanical quadratures formulae involving the
classical orthogonal polynomials, {\em Ann.\ Math.}\ 35 (1934),
658--677.
[States a general connection between Gauss--Jacobi quadrature
weights and the Lagrange polynomials.]
\parr
{\sc P. Wynn,} On a device for computing the $e_m(S_n)$
transformation, {\em Math.\ Comp.}\ 10 (1956), 91--96.
[Wynn's first of many papers about the epsilon algorithm for acceleration of
convergence of sequences.]
\par
{\sc S. Xiang and F. Bornemann,} On the convergence rates
of Gauss and Clenshaw--Curtis quadrature for functions of
limited regularity, archive 1203.2445v1, 2012.
\parr
{\sc K. Zhou, J. C. Doyle and K. Glover,} {\em Robust and Optimal Control},
Prentice Hall, 1996.
[A leading textbook on optimal control, with special attention
to approximation issues.]
\par
{\sc W. P. Ziemer,} {\em Weakly Differentiable Functions}, Springer, 1989.
[Includes a definition of
total variation in the measure theoretic context.]
\par

\end{par} \vspace{1em}



\end{document}
    
